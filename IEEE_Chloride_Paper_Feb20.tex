\documentclass[conference]{IEEEtran}
\IEEEoverridecommandlockouts
% The preceding line is only needed to identify funding in the first footnote. If that is unneeded, please comment it out.
%Template version as of 6/27/2024

\usepackage{placeins}

\usepackage{cite}
\usepackage{amsmath,amssymb,amsfonts}
\usepackage{algorithmic}
\usepackage{graphicx}
\usepackage{textcomp}
\usepackage{xcolor}
\def\BibTeX{{\rm B\kern-.05em{\sc i\kern-.025em b}\kern-.08em
    T\kern-.1667em\lower.7ex\hbox{E}\kern-.125emX}}
\begin{document}


\title{Comparison of Finite Difference and Finite Element Methods for Modeling Chloride Diffusion in Concrete}

\author{\IEEEauthorblockN{ Ramin Pahnabi}
\IEEEauthorblockA{\textit{Civil and Construction Engineering} \\
\textit{Brigham Young University}\\
Provo, United States  \\
rpahnabi@byu.edu}
\and
\IEEEauthorblockN{W. Spencer Guthrie}
\IEEEauthorblockA{\textit{Civil and Construction Engineering} \\
\textit{Brigham Young University}\\
Provo, United States \\
guthrie@byu.edu}
\and
\IEEEauthorblockN{Kendrick M. Shepherd}
\IEEEauthorblockA{\textit{Civil and Construction Engineering} \\
\textit{Brigham Young University}\\
Provo, United States \\
kendrick\_shepherd@byu.edu}
}


\maketitle

\begin{abstract}
\iffalse
Concrete is one of the most important and ubiquitous materials in the construction industry, and  is used for a wide range of applications: from large-scale infrastructure projects like bridges and dams to smaller-scale building projects like homes and commercial structures. 
Because concrete is brittle and has a relatively low tensile capacity compared to its compressive capacity, steel reinforcement is frequently introduced as a means of resisting the tensile forces that develop in the concrete cross-section. 
Consequently, the durability and functionality of a concrete structure is tightly coupled with that of its reinforcing steel rebar.
Particularly, chloride-induced corrosion and carbonation of concrete reinforcement poses a serious threat in reinforced concrete, and can result in structural degradation, safety issues, reduced lifespan, and financial losses. 
Numerical methods, like the Finite Difference Method (FDM) and the Finite Element Method (FEM), are widely used to predict chloride diffusion in concrete and its negative impact on rebar. 
In this study, we implement a FEM code and validate it against previously obtained FDM results (Dr. Guthrie's paper or results... this is a NIST software and there's a paper about it). 
The FEM result shows good agreement with the FDM. 
This study shows the acfcuracy of FEM and also its advantages in handling complex geometries.
\fi
A proper understanding of chloride ion ingress into concrete and effective mitigation of its effects are crucial for preventing corrosion and premature failure of reinforced concrete structures.
Previous National Institute of Standards and Technology(NIST) software employed the finite difference method (FDM) to evaluate chloride diffusion in concrete; however, this approach is not readily extensible to modeling complex geometries including the effects of reinforcing rebar. The purpose of this study is to develop finite element method (FEM) software that can exactly reproduce the FDM results, and can be extended to handle more general and complex geometries, such as concrete containing reinforcing steel (rebar).
The FEM results show good agreement with the FDM results and the FEM could be used for the more complicated problem indicating little difference between both methods.
\end{abstract}

\begin{IEEEkeywords}
Chloride, Finite Deference Method (FDM), Finite Element Method (FEM), Concrete, Corrosion
\end{IEEEkeywords}

%%%%%%%%%%%%%%%%%%%%%%%%%%%%%%%%%%%%%%%%%%%%%%%%%%%%%%%%
\section{Introduction}

Chloride-induced corrosion is the main cause of deterioration of reinforced concrete (RC) structures in chloride-laden environments\cite{b1} and consequently, much research effort has been put towards the prediction of chloride-induced corrosion damage in RC structures\cite{b2,b3,b4,b5,b6,b7}. 
A considerable amount of research has been undertaken on the chloride-induced corrosion process, and different types of models, including empirical, mathematical, numerical and probabilistic ones, have been developed for service life prediction of corrosion-affected RC structures\cite{b2,b3,b8}.
In the past, both empirical and numerical approaches have been used to develop corrosion rate prediction models, each with its own strengths and shortcomings. Experimental models work well within the test conditions they were created for, but they often do not apply effectively to situations beyond the experiments\cite{b9,b10,b11}. 
Therefore, numerical methods are frequently used to ensure reliable prediction in general conditions. These methods include the finite difference method (FDM) and finite element method(FEM). 

National Institute of Standards and Technology(NIST) historically deployed FDM software\cite{b18} used by state departments of transportation to predict chloride diffusion in concrete bridge decks\cite{b19}. While FDM provides a straightforward and computationally efficient method for modeling diffusion, it is often limited in handling complex geometries. 
There are many papers about the importance the FEM in chloride diffusion\cite{b12,b14}. Furthermore, studies \cite{b13}  show that the FEM can be used to predict crack initiation and propagation due to chloride-induced corrosion. These results then used to update chloride propagation and predictions to realistically represent how cracking and corrosion affect each other. \iffalse The following numerical FEM studies examined the influence of cracks in concrete exposed to chloride-rich environments, showing an increased corrosion rate of steel reinforcement due to macrocell interactions between corroding and passive areas \cite{b13,b14}.\fi

In this study, we leverage FEM's flexibility to simulate chloride diffusion and validate its performance against FDM results. While FDM is straightforward for simple domains, it struggles with the geometric complexity and boundary conditions. The variational formulation and capacity of FEM to operate unstructured meshed make it inherently more flexible. The following sections demonstrate how FEM provides alternative framework for predicting chloride penetration, ultimately supporting better durability evaluation and maintenance strategies for RC infrastructure.

Furthermore, a pathway is identified to translate the existing FDM code into the expanded FEM framework, enabling more accurate and flexible modeling of real-world structural conditions.


%%%%%%%%%%%%%%%%%%%%%%%%%%%%%%%%%%%%%%%%%%%%%%%%%%%%%%%%%
\section{Physics Problem}

\paragraph{Coupled system of equations}

Chloride ingress into concrete is often modeled using the coupled system of equations,
\begin{equation}
\begin{aligned}
\frac{\partial C_{\mathrm{free}}}{\partial t}
&=
\frac{\partial}{\partial x}
\left(
D(x)\,\frac{\partial C_{\mathrm{free}}}{\partial x}
\right)
-
k_{\rm react}\, r(t)\, C_{\mathrm{free}}\\
&\quad - k_{\mathrm{bind}}
\left(
\frac{\alpha C_{\mathrm{free}}}{1 + \beta C_{\mathrm{free}}}
-
C_{\mathrm{ads}}
\right)
+
f(t), \\
&\quad 0 < x < L,\; t > 0,
\end{aligned}
\label{eq:diff_react}
\end{equation}


\iffalse
\begin{equation}
\frac{\partial U}{\partial t}
=
\frac{\partial}{\partial x}
\left(
D(x)\,\frac{\partial U}{\partial x}
\right)
-
k_{\rm react}\, r(x)\, U
+
f(x),
\quad 0 < x < L,\; t > 0,
\label{eq:diff_react}
\end{equation}
\fi

\begin{equation}
\frac{dC_{\mathrm{react}}}{dt}
=
k_{\mathrm{react}}
r\left(
C_{\mathrm{react}}
\right)C_{\mathrm{free}},
\label{eq:reaction_rate}
\end{equation}


\begin{equation}
\frac{dC_{\mathrm{ads}}}{dt}
=
k_{\mathrm{bind}}
\left(
\frac{\alpha C_{\mathrm{free}}}{1 + \beta C_{\mathrm{free}}}
-
C_{\mathrm{ads}}
\right),
\label{eq:adsorption_rate}
\end{equation}


where $C_{\mathrm{free}}(x,t)$ denotes the free chloride concentration in the pore solution, $D(x)$ is the spatially varying diffusion coefficient, $k_{\mathrm{react}}$ is the reaction rate constant, $r(t)$ is a reaction indicator function that equals one in reactive regions and zero otherwise, and $f(t)$ represents a source term. The variable $C_{\mathrm{ads}}$ denotes the adsorbed chloride concentration (mol/m$^3$). The parameters $k_{\mathrm{bind}}$, $\alpha$, and $\beta$ control the binding kinetics and adsorption capacity, while $L$ denotes the length of the spatial domain.

The first equation represents the diffusion--reaction model governing the free chloride concentration. By neglecting the reaction term, the model reduces to the pure diffusion case (see (1)--(2)). Equation~(3) represents a Langmuir-type adsorption model, which describes the binding of chloride ions to the concrete. This model couples the diffusion of free chloride ions with an adsorption process. By accounting for this adsorption term, the bound chloride is separated from the total chloride content, and the remaining portion corresponds to the free chloride concentration.

\paragraph{Boundary Conditions}

The diffusion--reaction problem is closed by prescribing boundary conditions at both ends of the spatial domain. At the top boundary ($x=0$), the chloride concentration is typically fixed to a prescribed value, representing a constant or time--dependent exposure condition. This is enforced through a Dirichlet boundary condition,
\begin{equation}
U(0,t) = g,
\label{eq:bc_dirichlet}
\end{equation}
where $g$ denotes the imposed boundary concentration.

At the bottom boundary ($x=L$), chloride transport frequently is controlled through a specified outward flux, which is modeled using a Neumann boundary condition,
\begin{equation}
\left.
\frac{\partial U}{\partial x}
\right|_{x=L}
=
-\,h,
\label{eq:bc_neumann}
\end{equation}
where $h$ represents the prescribed flux at the boundary which is typically $0$. Together, these boundary conditions reflect a system in which chloride ingress is driven by exposure at one boundary while transport across the opposite boundary is regulated by a specified flux.

\iffalse
The coupled diffusion-reaction–adsorption problem is solved using a staggered time-integration scheme.\fi
%TODO reference


%%%%%%%%%%%%%%%%%%%%%%%%%%%%%%%%%%%%%%
\section{Numerical Methods}

The coupled diffusion--reaction--adsorption system is solved using a staggered (operator-splitting) time-integration scheme, in which diffusion and reaction/adsorption are solved sequentially at each time step~\cite{b17}. The Forward Euler method is used for all temporal discretization while we compute the spatial discretization with FEM.

\subsection{Finite Difference Method (FDM)}\label{AA}

 \paragraph{Spatial Discretization (Finite Difference)}

The spatial domain is discretized using a one-dimensional finite difference method. The interval $[0,L]$ is divided into $N$ uniform subintervals of size $\Delta x = L/N$, resulting in nodal points $x_i = i\,\Delta x$ for $i = 0,\dots,N$. The solution at each node is approximated by a discrete variable $d_i(t) \approx d(x_i,t)$.

Spatial derivatives in the diffusion term are approximated using second-order central differences.

%So, in this section, the FDM formulation is used for both spatial and time sections, but in the FEM, we have FEM in space and FDM in time.

Boundary conditions are incorporated directly into the discrete system. At the top boundary, the prescribed concentration is enforced by fixing the nodal value, while at the bottom boundary, the specified flux condition is imposed through a finite difference approximation of the spatial gradient. 

%%%%%%%%%%
%%%%%%%%%%
\iffalse
\paragraph{Spatial Discretization (Finite Difference)}

Divide the domain into $N$ intervals of size $\Delta x = L/N$, with nodes
$x_i = i\,\Delta x$, $i = 0,\dots,N$. Define $d_i(t) \approx d(x_i,t)$.

The standard central difference 1D FDM approximation for the diffusion term is:
$1 \le i \le N-1$,
\begin{equation}
\left.
\frac{\partial}{\partial x}
\left(
D\,\frac{\partial d}{\partial x}
\right)
\right|_{x_i}
\approx
\frac{
D_i (d_i - d_{i-1})
-
D_{i+1} (d_{i+1} - d_i)
}{
\Delta x^2
},
\label{eq:fdm_diff}
\end{equation}
where $d_i(t)$ denotes the nodal approximation of $d(x,t)$ at $x_i$,
$D_i = D(x_i)$ is the diffusion coefficient, and $\Delta x$ is the spatial
step size.

Dirichlet BC at $i=0$:
\begin{equation}
d_0(t) = g,
\label{eq:fdm_dirichlet}
\end{equation}
where $g$ is the prescribed boundary value.

Neumann BC at $i=N$ (forward difference approximation):
\begin{equation}
\frac{d_N - d_{N-1}}{\Delta x} = h,
\label{eq:fdm_neumann}
\end{equation}
which implies
\begin{equation}
d_{N-1} = d_N - h\,\Delta x.
\label{eq:fdm_neumann_elim}
\end{equation}
\fi
%%%%%%%%%%
%%%%%%%%%%

\paragraph{Time Discretization}

\textit{Forward Euler (Explicit) Method:} In the Forward Euler method, for the interior nodes we have, for $1 \le i \le N-1$,


\begin{equation}
\begin{aligned}
d_i^{n+1} 
&= d_i^n 
+ \Delta t \Bigl[
f_i
-
\frac{
D_i (d_i^n - d_{i-1}^n)
-
D_{i+1} (d_{i+1}^n - d_i^n)
}{\Delta x^2}
\\
&\qquad
-
k_{\mathrm{react}}\, r_i\, d_i^n
\Bigr], 
\end{aligned}
\label{eq:forward_euler_interior}
\end{equation}



With boundary conditions applied explicitly:
At the top surface of the concrete slab, the chloride concentration is assumed to be maintained at a fixed value, representing continuous exposure to a chloride-rich environment (considered as Dirichlet BCs). This condition enforces a constant chloride level at the exposed surface throughout the simulation.
At the bottom of the slab, chloride transport is governed by a prescribed flux condition (Neumann BCs) or a constant chloride level (Dirichlet BCs). This represents controlled exchange across the lower boundary, such as limited chloride penetration into an underlying layer. By specifying the rate at which chloride enters or leaves the slab at this boundary, the model captures the physical constraint on chloride movement at the base of the domain.


%%%%%%%%%%
\iffalse
In the top of the concrete slab, we have Dirichlet BC:
\begin{equation}
d_0^{n+1} = g,
\label{eq:dirichlet_time}
\end{equation}

Also, for the bottom of the slab, there is a Neumann BC:
\begin{equation}
\left.
\frac{\partial d}{\partial x}
\right|_{x=L}
=
h.
\label{eq:neumann_bc}
\end{equation}
\fi
%%%%%%%%%

Thus the explicit update at node $N$ is:
\begin{equation}
\begin{aligned}
d_N^{n+1}
&=
d_N^n
+
\Delta t
\Bigl[
f(x_N)
+
\frac{2h}{\Delta x}
+
\frac{2D_N}{\Delta x^2}
\bigl(d_{N-1}^n - d_N^n\bigr)
\\
&\qquad
-
k_{\mathrm{react}}\, r_N\, d_N^n
\Bigr],
\end{aligned}
\label{eq:forward_euler_boundary}
\end{equation}


\paragraph{CFL Condition for Explicit Scheme}\mbox{} To be sure that the explicit method (Forward-Euler) is reliable and the results  are stable, following limitation is considered:

\begin{equation}
{\rm CFL} = \frac{\max(D) \, \Delta t}{\Delta x^2} \le 0.5 \quad \text{for stability.}
\label{eq:cfl}
\end{equation}



%%%%%%%%%%%%%%%%%%%%%%%%%%%%%%%%%%%%%%%%%%%%%%%%%%%
%%%%%%%%%%%%%%%%%%%%%%%%% FEM %%%%%%%%%%%%%%%%%%%%%%%
%%%%%%%%%%%%%%%%%%%%%%%%%%%%%%%%%%%%%%%%%%%%%%%%%%%
\subsection{Finite Element Method (FEM)}\label{AA}

\paragraph{Spatial Discretization}

The one-dimensional diffusion--reaction problem is solved numerically using the
finite element method (FEM) with linear basis function. The formulation follows the standard Galerkin
approach described by \cite{b15}. FEM is particularly well suited for transport problems in heterogeneous materials such as concrete, as it naturally accommodates spatially varying material properties and allows for local mesh refinement \cite{b15}.

\iffalse Starting from the strong form of the governing equation, a weak (variational)
formulation is obtained by multiplying the equation by a suitable test function
and integrating over the spatial domain. Applying integration by parts to the
diffusion term reduces the regularity requirements on the solution and introduces
boundary contributions with a clear physical interpretation in terms of fluxes.\fi
In abstract form, the weak problem can be written as: find $u \in S$ such that
\begin{equation}
a(u,w) = \ell(w)
\qquad \forall w \in V ,
\end{equation}

\begin{equation}
\begin{aligned}
a(u,w) &=
\int_{\Omega}
D\, u,_x\, w,_x \, dx,
\iffalse+
\int_{\Omega}
R\, u\, w \, dx,
\fi
\end{aligned}
\label{eq:nitsche-bilinear-reaction}
\end{equation}


\begin{equation}
\begin{aligned}
\ell(w) &=
\int_{\Omega}
f\, w \, dx,
\end{aligned}
\label{eq:nitsche-linear-reaction}
\end{equation}


where $S$ and $V$ denote appropriate trial and test spaces, $a(\cdot,\cdot)$ is
the bilinear form associated with diffusion and reaction processes, and
$\ell(\cdot)$ represents the contribution of source terms and boundary fluxes \cite{b15}.

The solution field is approximated within a finite-dimensional subspace spanned
by shape functions defined over a partition of the spatial domain.
This leads to a discrete representation of the form:
\begin{equation}
u_h(x) = \sum_{i=1}^{N} d_i\, N_i(x),
\end{equation}
where $N_i(x)$ are the finite element shape functions and $d_i$ are the unknown
nodal values(coefficients). Substituting this approximation into the weak form and choosing the test functions from the same space yields a system of algebraic equations of the form.

The Standard form is:
\begin{equation}
\mathbf{K}\mathbf{d} = \mathbf{F},
\end{equation}
where $\mathbf{K}$ is the global stiffness matrix arising from diffusion and
reaction terms, $\mathbf{d}$ is the vector of nodal unknowns representing the
chloride concentration ($\mathbf{C_{free}}$), and $\mathbf{F}$ contains the contributions from source
terms and boundary fluxes. This system is then
solved to obtain the spatial distribution of chloride concentration.


%%%%%previous math section%%%%%%%%%%%
%%%%%previous math section%%%%%%%%%%%
%%%%%previous math section%%%%%%%%%%%
\iffalse
\subsection{Finite Element Method (FEM)}\label{AA}
\paragraph{Strong Form of one-Dimensional Diffusion--Reaction Problem}

Consider the one-dimensional diffusion--reaction equation defined in the physics problem section.

\paragraph{Weak Form}

Let $w(x)$ be a test function such that $w(0)=0$, consistent with the Dirichlet
condition \eqref{eq:bc-dirichlet}. Multiplying \eqref{eq:strong} by $w$ and
integrating over $\Omega$ yields
\begin{equation}
\int_{\Omega} D(x)\, u'(x)\, w'(x)\, dx
=
\int_{\Omega} f(x)\, w(x)\, dx
+ D(L)\, u'(L)\, w(L),
\label{eq:weak-raw}
\end{equation}

Using the Neumann condition \eqref{eq:bc-neumann}, the weak form becomes
\begin{equation}
\int_{\Omega} D(x)\, u'(x)\, w'(x)\, dx
=
\int_{\Omega} f(x)\, w(x)\, dx
- h\, w(L),
\label{eq:weak}
\end{equation}

Dirichlet boundary conditions are incorporated through a standard lifting
procedure, which modifies the load vector by
\begin{equation}
\mathbf{f}^{\rm Dirichlet}
=
- g\, \mathbf{k}_e(:,b),
\label{eq:dirichlet-lifting}
\end{equation}
where $b$ denotes the index of the boundary node.

\paragraph{Finite Element Approximation}

The finite element approximation of the trial and test functions is written as
\begin{equation}
u_h(x) = \sum_{b=0}^{p} U_b\, N_b(\xi),
\qquad
w_h(x) = \sum_{a=0}^{p} W_a\, N_a(\xi),
\label{eq:fe-approx}
\end{equation}
where $N_a(\xi)$ are Lagrange basis functions of degree $p$ defined on the
reference element $\xi \in [-1,1]$, and $U_b$ and $W_a$ are the nodal coefficients.

The mapping from reference to physical coordinates is given by
\begin{equation}
x = x(\xi),
\qquad
J = \frac{dx}{d\xi},
\label{eq:mapping}
\end{equation}

\paragraph{Element Stiffness Matrix}

The element stiffness matrix associated with diffusion is defined as
\begin{equation}
k^{(e)}_{ab}
=
\int_{-1}^{1}
D^{(e)}
\left( \frac{dN_a}{dx} \right)
\left( \frac{dN_b}{dx} \right)
J\, d\xi,
\label{eq:ke-def}
\end{equation}
where $D^{(e)}$ denotes the diffusion coefficient evaluated on element $e$.

Using the chain rule,
\begin{equation}
\frac{dN_a}{dx}
=
\frac{1}{J}\, \frac{dN_a}{d\xi},
\label{eq:chain-rule}
\end{equation}
\eqref{eq:ke-def} can be written in computable form as
\begin{equation}
k^{(e)}_{ab}
=
\int_{-1}^{1}
D^{(e)}
\frac{dN_a}{d\xi}
\frac{dN_b}{d\xi}
\frac{1}{J}\, d\xi,
\label{eq:ke-computable}
\end{equation}

\paragraph{Element Force Vector}

The element load vector corresponding to the body force is
\begin{equation}
f^{(e)}_{a}
=
\int_{-1}^{1}
N_a(\xi)\,
f\!\left( x(\xi) \right)
J\, d\xi,
\label{eq:fe}
\end{equation}

\paragraph{Boundary Contributions}

For a Neumann boundary condition applied at $x=L$, the corresponding boundary
load contribution is
\begin{equation}
f^{\rm Neumann}_{a}
=
h\, N_a(x=L),
\label{eq:neumann-right}
\end{equation}
An analogous expression holds at $x=0$ if a Neumann condition is prescribed
there.

Dirichlet boundary conditions are enforced via the lifting term
\eqref{eq:dirichlet-lifting}, which modifies the global right-hand side during
assembly.

\paragraph{Global System}

After assembling all element contributions, the global discrete system takes
the form
\begin{equation}
\mathbf{K}\, \mathbf{U} = \mathbf{F},
\label{eq:global-system}
\end{equation}
where $\mathbf{K}$ is the global stiffness matrix, $\mathbf{U}$ is the vector of
unknown nodal values, and $\mathbf{F}$ is the global load vector.
\fi
%%%%%%%%previous math section%%%%%%%
%%%%%%%%previous math section%%%%%%%
%%%%%%%%previous math section%%%%%%%


\paragraph{Time Discretization}

For time-dependent problems, the semi-discrete finite element formulation
leads to a system of ordinary differential equations of the form
\begin{equation}
\mathbf{M}\, \dot{\mathbf{d}}(t)
+
\mathbf{K}\, \mathbf{d}(t)
+
\mathbf{RM}\, \mathbf{d}(t)
=
\mathbf{F}(t),
\label{eq:semi-discrete}
\end{equation}
where $\mathbf{M}$ is the mass matrix, $\mathbf{K}$ is the diffusion stiffness
matrix, $\mathbf{R}$ is the reaction matrix, $\mathbf{d}(t)$ is the vector of
time-dependent nodal unknowns, and $\mathbf{F}(t)$ is the load vector.


To account for limited binding capacity, the reaction term is controlled by a binary flag $r \in \{0,1\}$. The flag is defined as
\[
r =
\begin{cases}
1, & C_{free} > C_{react}, \\
0, & C_{free} \le C_{react},
\end{cases}
\]
where $C_{free}$ and $C_{react}$ denote the free and reacted (bound) chloride concentrations, respectively. When the reacted chloride level reaches or exceeds the free chloride, further reaction is deactivated.

The semi-discrete system becomes
\begin{equation}
\mathbf{M}\dot{\mathbf{d}}(t)
+ \mathbf{K}\mathbf{d}(t)
+ r\,\mathbf{M} \mathbf{d}(t)
= \mathbf{F}(t).
\end{equation}


\textit{Forward Euler Scheme}: Using a Forward Euler discretization in time, the time derivative in
\eqref{eq:semi-discrete} is approximated as
\[
\dot{\mathbf{d}}(t^n)
\approx
\frac{\mathbf{d}^{n+1} - \mathbf{d}^n}{\Delta t},
\]
where $\Delta t$ denotes the time-step size and the superscript $n$ indicates
the discrete time level.

Substituting this approximation into \eqref{eq:semi-discrete} and solving for
$\mathbf{d}^{n+1}$ yields the explicit update formula
\begin{equation}
\mathbf{d}^{n+1}
=
\mathbf{d}^n
+
\Delta t \, \mathbf{M}^{-1}
\big(
\mathbf{F}^n
-
\mathbf{K}\, \mathbf{d}^n
-
\mathbf{rM}\, \mathbf{d}^n
\big),
\label{eq:forward-euler}
\end{equation}
which advances the solution from time level $t^n$ to $t^{n+1}$.

Equation \eqref{eq:forward-euler} is conditionally stable and requires the time
step $\Delta t$ to satisfy a problem-dependent stability constraint \eqref{eq:cfl}.

\iffalse
Equation \eqref{eq:forward-euler} is conditionally stable and requires the time
step $\Delta t$ to satisfy a stability constraint of the form
\[
\Delta t \le \frac{2}{\lambda_{\max}(M^{-1}K)},
\]
where $\lambda_{\max}(M^{-1}K)$ is the largest eigenvalue of the discrete system.
\fi


%%%%%%%%%%%%%%%%%%%%%%%%%%%NITSCHE%%%%%%%%%%%%%%%%%%%%%%%
%%%%%%%%%%%%%%%%%%%%%%%%%%%NITSCHE%%%%%%%%%%%%%%%%%%%%%%%
%%%%%%%%%%%%%%%%%%%%%%%%%%%NITSCHE%%%%%%%%%%%%%%%%%%%%%%%
\subsection{Nitsche Method for Weak Imposition of Dirichlet Boundary Conditions}

To weakly impose essential boundary conditions within the finite element
framework, Nitsche’s method is employed \cite{b16}. This approach allows Dirichlet boundary conditions to be enforced variationally, while maintaining consistency and stability of the numerical scheme. In contrast, finite difference methods (FDM) apply boundary conditions directly at grid points, making their implementation simple, especially for structured domains.

\iffalse We consider the time-dependent diffusion--reaction problem in \eqref{eq:diff_react}.\fi In Nitsche’s method \cite{b16}, the standard weak formulation is augmented by additional boundary terms that weakly enforce the Dirichlet condition. These terms ensure consistency with the strong form, restore symmetry of the bilinear form, and add a stabilization contribution controlled by a penalty parameter.

\iffalse
For transient problems, Nitsche’s formulation is applied at each time level,
and the resulting semi-discrete system is subsequently advanced in time using
the chosen time integration scheme. This separation allows the weak imposition
of boundary conditions to remain entirely within the spatial discretization,
while time dependence enters naturally through the temporal derivative.
\fi

The resulting bilinear form is

\begin{equation}
\begin{aligned}
a_{\rm Nitsche}(u,w)
&=
a(u,w) -
\int_{\Gamma_D}
D\, u,_x\, w \, ds\\
&\quad
-
\int_{\Gamma_D}
D\, w,_x\, u \, ds
+
\int_{\Gamma_D}
\frac{\gamma D}{h_e}\, u\, w \, ds,
\end{aligned}
\label{eq:nitsche-bilinear-reaction}
\end{equation}
where $u$ is the trial function, $w$ is the test function, $h_e$ is a
characteristic boundary element length, and $\gamma > 0$ is a penalty parameter. A value of $\gamma = 100$ was used, which yielded results in agreement with the NIST software.

The associated linear functional is given by


\begin{equation}
\begin{aligned}
\ell_{\rm Nitsche}(w)
&=
\ell(w) + 
\int_{\Gamma_D}
D\, w,_x\, g \, ds
+
\int_{\Gamma_D}
\frac{\gamma D}{h_e}\, g\, w \, ds,
\end{aligned}
\label{eq:nitsche-linear-reaction}
\end{equation}
where $f$ is the body force and $g$ denotes the prescribed Dirichlet boundary
value.

%%%%%%%%%
\iffalse
The reaction term contributes only through the domain integral in
\eqref{eq:nitsche-bilinear-reaction} and does not require additional boundary
terms, as it does not involve spatial derivatives.
\fi
%%%%%%%%%

\paragraph{Element-Level Contributions}

\iffalse
At the element level, the diffusion and reaction contributions to the stiffness
matrix are
\begin{equation}
K^{(e)}_{ab}
=
\int_{\Omega_e}
\left(
D\, N_a' N_b'
+
R\, N_a N_b
\right)
dx,
\label{eq:bulk-ke-reaction}
\end{equation}
where $N_a$ and $N_b$ are local basis functions and primes denote derivatives
with respect to the physical coordinate $x$.
\fi

For elements adjacent to $\Gamma_D$, the Nitsche boundary terms add the
following contributions to the stiffness matrix:
\begin{equation}
K^{(e)}_{ab}
\mathrel{+}=
\int_{\Omega_e}
D
\left(
- N_a' N_b
- N_b' N_a
+ \frac{\gamma}{h_e} N_a N_b
\right)
dx,
\label{eq:nitsche-ke-reaction}
\end{equation}
and to the load vector:
\begin{equation}
F^{(e)}_a
\mathrel{+}=
\int_{\Omega_e}
\left(
-
D\, N_a'\, g
+
\frac{\gamma D}{h_e}\, N_a\, g
\right)
dx,
\label{eq:nitsche-fe-reaction}
\end{equation}


%NEW IMAGES
\section{Validation and Comparison}

Here we present a comparison between the finite difference method (FDM), the standard finite element method (FEM) without Nitsche’s method and without mass lumping, and a modified FEM incorporating Nitsche’s method and mass lumping. 

Without the use of the Nitsche method and mass lumping, the finite element
results were found to be in close agreement with those obtained using the
finite difference method (FDM). In the Nitsche formulation, Dirichlet boundary
conditions are imposed weakly through the introduction of a penalty term
parameterized by $\gamma$, rather than being enforced strongly. 

When the Nitsche method is combined with mass lumping, the resulting finite
element solution reproduces the FDM results exactly. This agreement provides
confidence that the proposed numerical techniques do not alter the underlying
physics of the problem and remain consistent with the physics-based simulation
benchmarks defined by the National Institute of Standards and Technology
(NIST) software.

In FEM(-ML +N), ML refers to mass lumping and N refers to Nitsche’s method. A plus sign (+) indicates that the corresponding technique is used, while a minus sign (-) indicates that it is not. Using this notation, the FEM variants are as follows:  
\begin{itemize}
    \item FEM(-ML -N): without mass lumping and without Nitsche’s method  
    \item FEM(-ML +N): without mass lumping but with Nitsche’s method  
    \item FEM(+ML -N): with mass lumping but without Nitsche’s method  
    \item FEM(+ML +N): with mass lumping and with Nitsche’s method
\end{itemize}

Fig.~\ref{fig:fdm-vs-fem-nitsche_100} shows the results on a coarse mesh, while Fig.~\ref{fig:fdm-vs-fem-nitsche_500} presents the corresponding comparison for a fine mesh. \iffalse Each figure contains five curves: FDM, FEM(-ML -N), FEM(+ML -N), FEM(-ML +N), and FEM(+ML +N).\fi



\begin{figure}[!b]
\centering
\includegraphics[width=\linewidth]{100_plots.png}
\caption{Comparing FDM with FEM with Nitsche and ML with coarse mesh.}
\label{fig:fdm-vs-fem-nitsche_100}
\end{figure}

\begin{figure}[!tb]
\centering
\includegraphics[width=\linewidth]{500_plots.png}
\caption{Comparing FDM with FEM with Nitsche and ML with fine mesh.}
\label{fig:fdm-vs-fem-nitsche_500}
\end{figure}

The next set of figures illustrates the differences between the FEM solutions
and the reference FDM solution.


\begin{figure}[!tb]
\centering
\includegraphics[width=\linewidth]{100_error.png}
\caption{Error comparison between FDM and FEM formulations on the coarse mesh.}
\label{fig:fdm-vs-fem-error-all-coarse}
\end{figure}

\begin{figure}[!tb]
\centering
\includegraphics[width=\linewidth]{500_error.png}
\caption{Error comparison between FDM and FEM formulations on the fine mesh.}
\label{fig:fdm-vs-fem-error-all-fine}
\end{figure}


Fig.~\ref{fig:fdm-vs-fem-error-all-coarse} compares the FDM solution with all of the four FEM formulations.
The Fig.~\ref{fig:fdm-vs-fem-error-all-coarse} plot corresponds to the coarse mesh, while Fig.~\ref{fig:fdm-vs-fem-error-all-fine}  the second corresponds to
the fine mesh.

As observed in the previous figures, the discrepancy between the FDM solution
and the FEM formulation using Nitsche’s method without mass lumping is
significantly larger than the others. To improve clarity and enable a more
meaningful comparison among the remaining methods, this formulation is omitted
in the following plots.

\begin{figure}[!tb]
\centering
\includegraphics[width=\linewidth]{100_error_without_NwoM.png}
\caption{Error comparison between FDM and selected FEM formulations on the coarse mesh.}
\label{fig:fdm-vs-fem-error-reduced-coarse}
\end{figure}

\begin{figure}[!tb]
\centering
\includegraphics[width=\linewidth]{500_error_without_NwoM.png}
\caption{Error comparison between FDM and selected FEM formulations on the fine mesh.}
\label{fig:fdm-vs-fem-error-reduced-fine}
\end{figure}

Accordingly, Fig.~\ref{fig:fdm-vs-fem-error-reduced-coarse} presents the comparison
between the FDM solution and the three FEM formulations, excluding FEM(-ML +N). 
Fig.~\ref{fig:fdm-vs-fem-error-reduced-coarse} corresponds to the coarse mesh, 
while Fig.~\ref{fig:fdm-vs-fem-error-reduced-fine} shows the results for the fine mesh.

Accordingly, we compare the coarse mesh (100 spatial layers, 500 time steps) and the fine mesh (500 spatial layers, 100\,000 time steps) to assess the influence of mesh resolution on the solution. 
We consider the FDM solution and four FEM formulations: 
(i)~FEM(-ML -N), 
(ii)~FEM(+ML -N), 
(iii)~FEM(-ML +N), and 
(iv)~FEM(+ML +N).

%%%%%%
\begin{table}[!tb]
\caption{Convergence between coarse and fine meshes: maximum and mean absolute difference of the solution \iffalse(fine minus coarse interpolated onto the fine grid)\fi.}
\label{tab:mesh-convergence}
\centering
\begin{tabular}{|l|c|c|}
\hline
\textbf{Method} & \textbf{Max.\ abs.\ diff.} & \textbf{Mean abs.\ diff.} \\
\hline
FDM & 0.0389 & 0.00412 \\
\hline
FEM(-ML -N) & 0.0445 & 0.00569 \\
\hline
FEM(+ML -N) & 0.0440 & 0.00475 \\
\hline
FEM(-ML +N) & 0.201 & 0.0281 \\
\hline
FEM(+ML +N) & 0.0389 & 0.00412 \\
\hline
\end{tabular}
\end{table}

Table~\ref{tab:mesh-convergence} reports the maximum and mean absolute differences between the fine-mesh solution and the coarse-mesh solution (interpolated onto the fine grid).
All methods exhibit convergence with mesh refinement.
For FDM and the standard FEM formulations (with or without mass lumping), the discrepancies are of order $10^{-2}$ in the maximum norm and about $5\times10^{-3}$ on average, indicating that the coarse mesh already provides a reasonably accurate approximation.

A larger difference is observed for the FEM formulation using Nitsche's method without mass lumping (maximum difference $\approx 0.20$, mean $\approx 0.028$).
This indicates a stronger sensitivity of this formulation to mesh resolution.
When mass lumping is combined with Nitsche's method, however, the differences reduce to the same level as the other methods.
Overall, these results confirm the convergence of all approaches while highlighting the improved stability and mesh robustness obtained by combining Nitsche's method with mass lumping.

\iffalse
Accordingly, we compare the coarse mesh (100 spatial layers, 500 time steps) and the fine mesh (500 spatial layers, 100\,000 time steps) to assess the influence of mesh size on the solution.
Again we consider the FDM solution and four FEM formulations:
(i)~FEM without Nitsche's method and without mass lumping,
(ii)~FEM without Nitsche's method and with mass lumping,
(iii)~FEM with Nitsche's method and without mass lumping, and
(iv)~FEM with both Nitsche's method and mass lumping.
Table~\ref{tab:mesh-convergence} reports the maximum and mean absolute difference between the fine-mesh solution and the coarse-mesh solution (interpolated onto the fine grid).
The largest differences occur for FEM with Nitsche and without mass lumping (max.\ abs.\ diff.\ $\approx 0.20$, mean $\approx 0.028$); the other formulations show much smaller differences (max.\ $\approx 0.04$, mean $\approx 0.005$), indicating that the fine mesh is particularly valuable when using Nitsche's method without mass lumping.
\fi
%%%%%%


The same domain and parameters are considered for FDM and FEM. Some of them are listed in Table~\ref{tab:model-parameters}.


%%%%%%Appendix
\begin{table}[!tb]
\caption{Model parameters used in the chloride ingress simulations}
\label{tab:model-parameters}
\centering
\begin{tabular}{|l|l|}
\hline
\textbf{Symbol} & \textbf{Description} \\
\hline
$t_{\max}$ &
\begin{tabular}[c]{@{}l@{}}
Maximum number of days of exposure: 1800 days
\end{tabular} \\
\hline
$d$ &
\begin{tabular}[c]{@{}l@{}}
Thickness of the specimen: 0.25 m
\end{tabular} \\
\hline
$WC$ &
\begin{tabular}[c]{@{}l@{}}
Water--cement ratio: 0.5
\end{tabular} \\
\hline
$hy$ &
\begin{tabular}[c]{@{}l@{}}
Degree of hydration: 0.75
\end{tabular} \\
\hline
$VF$ &
\begin{tabular}[c]{@{}l@{}}
Volume fraction of aggregate: 70\%
\end{tabular} \\
\hline
$AC$ &
\begin{tabular}[c]{@{}l@{}}
Air content (volume fraction): 2.0\%
\end{tabular} \\
\hline
$D_i$ &
\begin{tabular}[c]{@{}l@{}}
Time-dependent diffusivity: 6e-12
\end{tabular} \\
\hline
$\theta_i$ &
\begin{tabular}[c]{@{}l@{}}
Exterior temperature at month $i$ of exposure:
Table~\ref{tab:monthly-data}.
\end{tabular} \\

\iffalse
\hline
$\theta_{\max}$ &
\begin{tabular}[c]{@{}l@{}}
Maximum temperature: 303.15 F
\end{tabular} \\
\hline
$\theta_{\mathrm{cur}}$ &
\begin{tabular}[c]{@{}l@{}}
Temperature of current month
\end{tabular} \\
\hline
$M_i$ &
\begin{tabular}[c]{@{}l@{}}
Molarity at month $i$ of exposure
\end{tabular} \\
\hline
$M_{\mathrm{cur}}$ &
\begin{tabular}[c]{@{}l@{}}
Molarity of current month
\end{tabular} \\
\fi

\hline
$t_{\mathrm{cur}}$ &
\begin{tabular}[c]{@{}l@{}}
Time before exposure begins (s): 28 days
\end{tabular} \\
\hline
$\kappa_{\mathrm{rel}}^{\mathrm{surf}}$ &
\begin{tabular}[c]{@{}l@{}}
Relative diffusivity for surface layer: 1
\end{tabular} \\
\hline
$d_{\mathrm{surf}}$ &
\begin{tabular}[c]{@{}l@{}}
Thickness of the surface layer: 5.0 mm
\end{tabular} \\
\hline
$E$ &
\begin{tabular}[c]{@{}l@{}}
Activation energy for chloride diffusion: 40 kJ/mole 
\end{tabular} \\
\hline
$R$ &
\begin{tabular}[c]{@{}l@{}}
Universal gas constant: 0.008314 kJ/(mol K)
\end{tabular} \\

\iffalse
\hline
$t_{\mathrm{fac}}$ &
\begin{tabular}[c]{@{}l@{}}
Relative Arrhenius factor at curing time: \\
$e^{-E/(\theta_{\mathrm{cur}}R)} / e^{-E/(298.15R)}$
\end{tabular} \\
\hline
$t_{\max}$ &
\begin{tabular}[c]{@{}l@{}}
Maximal Arrhenius factor at $\theta_{\max}$ (K): \\
$e^{-E/(\theta_{\max}R)} / e^{-E/(298.15R)}$
\end{tabular} \\
\fi

\hline
$\alpha$ &
\begin{tabular}[c]{@{}l@{}}
Binding isotherm parameter: 1.67
\end{tabular} \\
\hline
$\beta$ &
\begin{tabular}[c]{@{}l@{}}
Binding isotherm parameter: 4.08
\end{tabular} \\

\iffalse
\hline
$k_{\mathrm{bind}}$ &
\begin{tabular}[c]{@{}l@{}}
Binding rate constant; capped at $0.2/\Delta t$ \\
to satisfy CFL stability for the reaction term
\end{tabular} \\
\fi

\hline
$k_{\mathrm{bind}}$ &
\begin{tabular}[c]{@{}l@{}}
Binding rate constant: 1e-07
\end{tabular} \\
\hline
$C3A$ &
\begin{tabular}[c]{@{}l@{}}
C3A volume fraction (typically 5--10\%)
\end{tabular} \\
\hline
$C4AF$ &
\begin{tabular}[c]{@{}l@{}}
C4AF volume fraction (typically 5--13\%)
\end{tabular} \\
\hline
$C3A_{\mathrm{react}}$ &
\begin{tabular}[c]{@{}l@{}}
Formation of Friedel's salt from C3A (= 7.419)
\end{tabular} \\
\hline
$C4AF_{\mathrm{react}}$ &
\begin{tabular}[c]{@{}l@{}}
Formation of Friedel's salt from C4AF (= 4.119)
\end{tabular} \\

\iffalse
\hline
$k_{\mathrm{react}}$ &
\begin{tabular}[c]{@{}l@{}}
Chloride--aluminate reaction rate; capped at \\
$0.2/\Delta t$ to satisfy CFL stability
\end{tabular} \\
\fi

\hline
$k_{\mathrm{react}}$ &
\begin{tabular}[c]{@{}l@{}}
Chloride--aluminate reaction rate: 1e-08
\end{tabular} \\
\hline

\iffalse
$t_{\mathrm{appl}}$ &
\begin{tabular}[c]{@{}l@{}}
Surface treatment application time
\end{tabular} \\
\hline
$r_{\mathrm{flags}}$ &
\begin{tabular}[c]{@{}l@{}}
Indicator for activation of reaction terms
\end{tabular} \\
\hline
$\gamma_{\mathrm{cem}}$ &
\begin{tabular}[c]{@{}l@{}}
Specific gravity of cement
\end{tabular} \\
\hline
$PF$ &
\begin{tabular}[c]{@{}l@{}}
Paste fraction $(1 - VF - AC)$
\end{tabular} \\
\hline
$Por$ &
\begin{tabular}[c]{@{}l@{}}
Pore liquid volume fraction: \\
$PF \cdot \dfrac{WC - 0.23\,hy}{WC + 1/\gamma_{\mathrm{cem}}}$ \\
(numerator clamped to be non-negative)
\end{tabular} \\
\hline
$MM_{\mathrm{Cl}}$ &
\begin{tabular}[c]{@{}l@{}}
Molecular mass of chloride
\end{tabular} \\
\hline
$\phi_0$ &
\begin{tabular}[c]{@{}l@{}}
Initial chloride concentration, modified as \\
$\phi_0 = \phi_0 \dfrac{PF \cdot 1000 \cdot \gamma_{\mathrm{cem}}}
{(1 + WC \gamma_{\mathrm{cem}})\, Por \, MM_{\mathrm{Cl}}}$
\end{tabular} \\
\hline
$\kappa_{\mathrm{conc}}$ &
\begin{tabular}[c]{@{}l@{}}
Diffusivity of concrete base layer: \\
$\kappa_\infty + D_i \, t_{\mathrm{cure}}^{-D_M}$
\end{tabular} \\
\hline
$\kappa_{\mathrm{skin}}$ &
\begin{tabular}[c]{@{}l@{}}
Diffusivity of surface layer: \\
$\kappa_{\mathrm{conc}} \cdot \kappa_{\mathrm{rel}}^{\mathrm{surf}}$
\end{tabular} \\
\hline
$\Delta d$ &
\begin{tabular}[c]{@{}l@{}}
Spatial element thickness
\end{tabular} \\
\hline
$\Delta t$ &
\begin{tabular}[c]{@{}l@{}}
Time step: $\Delta t = 0.2 (\Delta d)^2 /
\max\{\kappa_{\mathrm{conc}}, \kappa_{\mathrm{skin}}\}$ \\
(CFL condition accounting for aging and\\
temperature effects)\\
\end{tabular} \\
\hline
$Cl_{\max}$ &
\begin{tabular}[c]{@{}l@{}}
Maximum chloride binding capacity: \\
$\frac{(C3A \cdot C3A_{\mathrm{react}} + C4AF \cdot C4AF_{\mathrm{react}}}{(WC - 0.23\,hy)})$
\end{tabular} \\
\hline
\fi

\end{tabular}
\end{table}

\begin{table}[!tb]
\caption{Monthly chloride concentration and temperature values used in the simulation.}
\label{tab:monthly-data}
\centering
\begin{tabular}{|l|c|c|}
\hline
\textbf{Month} & \textbf{Cl$^{-}$} & \textbf{Temperature (K)} \\
\hline
0  & 4.0000 & 278.15 \\
\hline
1  & 4.0000 & 278.15 \\
\hline
2  & 4.0000 & 288.15 \\
\hline
3  & 4.0000 & 288.15 \\
\hline
4  & 4.0000 & 293.15 \\
\hline
5  & 4.0000 & 298.15 \\
\hline
6  & 4.0000 & 303.15 \\
\hline
7  & 4.0000 & 303.15 \\
\hline
8  & 4.0000 & 298.15 \\
\hline
9  & 4.0000 & 293.15 \\
\hline
10 & 4.0000 & 288.15 \\
\hline
11 & 4.0000 & 283.15 \\
\hline
\end{tabular}
\end{table}



\iffalse
\begin{table}[!tb]
\caption{Simulation Parameters for FDM and FEM}
\label{tab:parameters}
\centering
\begin{tabular}{|l|c|}
\hline
\textbf{Parameter} & \textbf{Value} \\
\hline
Number of layers & 99 \\
\hline
Thickness of sample & 0.25 m \\
\hline
Boundary conditions & Dirichlet (top), Neumann (bottom) \\
\hline
Total duration of exposure & 1800 days \\
\hline
w/c ratio of concrete & 0.5 \\
\hline
Degree of hydration & 0.75 \\
\hline
Volume fraction of aggregate & 70\% \\
\hline
Initial chloride concentration & 0.0 g Cl/g cement \\
\hline
Air content & 2.0\% \\
\hline
Activation energy for diffusion & 40 kJ/mole \\
\hline
\end{tabular}
\end{table}
\fi

\iffalse
In the standard FEM formulation, neither the Nitsche's method nor mass lumping was employed. As shown in the results, the standard FEM solution does not exactly match the FDM solution: the values agree up to the third decimal place, while differences appear after that. However, after switching to the modified FEM formulation, incorporating the Nitsche's method for boundary enforcement and mass lumping, the FEM results match the FDM solution exactly.
\fi


\section{Discussion}

\subsection{Advantages of FEM over FDM}

The finite element method (FEM) offers several advantages over the finite difference method (FDM), particularly for engineering applications including complex geometries and boundary conditions. Unlike standard FDM, which is typically restricted to structured grids and regular domains, FEM can naturally accommodate irregular geometries and spatially varying material properties. In addition, FEM provides greater flexibility in the treatment of complex boundary conditions and is generally more suitable for solving problems involving complicated physical processes. As a result, complex problems can often be modeled more effectively using FEM than FDM.

\subsection{Limitations}

Despite their effectiveness, both FEM and FDM require careful treatment of boundary conditions and discretization parameters to ensure numerical accuracy and stability. The FDM is computationally efficient and simple to implement, but its applicability is limited for irregular domains and complex geometries. On the other hand, while FEM provides greater flexibility and modeling capability, it typically involves higher computational cost and increased implementation complexity compared to FDM.

%%%%%%%%%%
\iffalse
\section{Discussion}
\subsection{Advantages of FEM over FDM:} 
\begin{itemize}
	\item FEM can naturally handle irregular geometries, and complex boundary conditions, which are challenging for standard FDM.
    	\item The complex problem could be solved with the FEM better than FDM. 
\end{itemize}

\subsection{Limitations:}
\begin{itemize}
    	 \item Both FEM and FDM require careful treatment of boundary conditions and discretization to ensure accuracy. FDM is less flexible for irregular domains, while FEM may involve higher computational cost and implementation complexity.
\end{itemize}
     
\subsection{Potential extensions:} 
The current study can be extended to 2D and 3D simulations, space-time finite element formulations, coupling with mechanical degradation models, and phase-field modeling for more complex physical phenomena in reinforced concrete structures.
%%%%%%%%%%%%

\section{Conclusions}
\begin{itemize}
    \item FEM with proper boundary treatment (e.g., Nitsche's method and mass lumping) successfully reproduces FDM results for chloride diffusion in 1D.
    \item FEM provides greater flexibility for modeling complex problems in reinforced concrete, including irregular geometries.
    \item Future work: Extend the approach to higher dimensional problem, incorporate coupled mechanical-chemical degradation processes, and explore phase-field modeling for fracture and diffusion interactions.
\end{itemize}
\fi
%%%%%%%%%
\section{Conclusions}

In this study, the finite element method (FEM) was applied to simulate one-dimensional chloride diffusion and its performance was compared with the finite difference method (FDM). The results show that, with appropriate boundary treatment, such as the use of Nitsche's method and mass lumping, the FEM solutions closely reproduce the reference FDM results. The results of the FEM(+ML +N) are still comparable for the considered diffusion problem. \iffalse This confirms the accuracy and reliability of the FEM formulation for the considered diffusion problem.\fi

In addition to its accuracy, FEM offers greater flexibility for modeling complex engineering problems in reinforced concrete. In particular, it allows the treatment of irregular geometries, spatially varying material properties, and more sophisticated boundary conditions, which are difficult to handle using standard FDM approaches.

{Potential extensions}: The present study motivates the extension of the FEM analysis for chloride diffusion in many directions, including 2D and 3D analysis, alternative temporal discretization extension to cracking to advective flow.


\begin{thebibliography}{00}
\bibitem{b1}
M. Otieno, H. Beushausen, and M. Alexander, ``Chloride-induced corrosion of steel in cracked concrete—Part II: Corrosion rate prediction models," Cement and Concrete Research, Vol. 41, No. 4, pp. 443--452, 2011.
\bibitem{b2}U. M. Angst, E. Rossi, C. Boschmann K\"athler, D. Mannes, P. Trtik,
B. Elsener, Z. Zhou, and M. Strobl, ``Chloride-induced corrosion of steel in concrete—insights from bimodal neutron and X-ray microtomography combined with ex-situ microscopy," Materials and Structures, 2024.
\bibitem{b3}M. Otieno, H. Beushausen, and M. Alexander, ``Chloride-induced corrosion of steel in cracked concrete—Part I: Experimental studies under accelerated and natural marine environments,"
Cement and Concrete Research, 2015.
\bibitem{b4}
C. S. Das, H. Zheng, and J.-G. Dai, ``A review of chloride-induced steel corrosion in coastal reinforced concrete structures: Influence of micro-climate," Construction and Building Materials, 2024.
\bibitem{b5}
C. Liang, Z. Cai, H. Wu, J. Xiao, Y. Zhang, and Z. Ma, ``Chloride transport and induced steel corrosion in recycled aggregate concrete: A review," Construction and Building Materials, 2021.
\bibitem{b6}J. N\v{e}me\v{c}ek, P. Tr\'avn\'i\v{c}ek, J. N\v{e}me\v{c}kov\'a, and J. Kruis, ``Mitigation of chloride induced corrosion in reinforced concrete structures and its modeling," Computers and Concrete, 2021.  
\bibitem{b7}
S. C. Paul and G. P. A. Greeff van Zijl, ``Corrosion deterioration of steel in cracked SHCC," International Journal of Concrete Structures and Materials, 2017.
\bibitem{b8} J. Zhang and M. M. S. Cheung, ``Modeling of chloride-induced corrosion in reinforced concrete structures," Materials and Structures, Vol. 45, No. 10, pp. 1555--1566, 2012.
\bibitem{b9} P. Li, C. Li, C. Jia, and D. Li, ``A comparative study on chloride diffusion in concrete exposed to different marine environment conditions," Cement and Concrete Composites, 2024.  
\bibitem{b10} P. Su, Q. Dai, and E. S. Kane, ``Predicting chloride ingression in concrete containing different SCMs based on chloride binding and electrical resistivity," Construction and Building Materials, 2024.  
\bibitem{b11} T. Ferenc, E. Wojtczak, B. Meronk, J. Ryl, K. Wilde, and M. Rucka, ``Characterization of corrosion-induced fracture in reinforced concrete beams using electrical potential, ultrasound and low-frequency vibration," Construction and Building Materials, 2023. 
\bibitem{b12}J. Xiao, J. Ying, and L. Shen, ``FEM simulation of chloride diffusion in modeled recycled aggregate concrete," Construction and Building Materials, Vol.~29, pp.~12--23, 2012. 
\bibitem{b13}
E. Redaelli, L. Bertolini, W. Peelen, and R. Polder, ``FEM-models for the propagation period of chloride induced reinforcement corrosion," Materials and Corrosion, Vol. 57, No. 8, pp. 628--635, 2006.
\bibitem{b14} E. Korec, M. Jir\'asek, H. S. Wong, and E. Mart\'inez-Pa\~neda, ``A phase-field chemo-mechanical model for corrosion-induced cracking in reinforced concrete," Mechanics of Materials, 2023.
\bibitem{b15} T.~J.~R. Hughes, The Finite Element Method: Linear Static and Dynamic Finite Element Analysis. Upper Saddle River, NJ: Prentice Hall, 2000.
\bibitem{b16} E.~Burman, ``A consistent Nitsche formulation for the weak imposition
of boundary conditions in convection--diffusion problems,'' Computer Methods in Applied Mechanics and Engineering, vol.~199, no.~47--48, pp.~2845--2855, 2010.
\iffalse \bibitem{b17} {\'A}.~Havasi and J.~Bartholy, ``Splitting method and its application in air pollution modeling,'' Id\H{o}j\'ar\'as, vol.~105, no.~1, pp.~39--58, 2001.\fi
\bibitem{b17} D.~Lanser and J.~G.~Verwer, ``Analysis of operator splitting for advection--diffusion--reaction problems from air pollution modelling,'' Journal of Computational and Applied Mathematics, vol.~111, no.~1--2, pp.~201--216, 1999.
\bibitem{b18} A. W. Birdsall, W. S. Guthrie, and D. P. Bentz, ``Effects of initial surface treatment timing on chloride concentrations in concrete bridge decks,'' Transp. Res. Rec., vol. 2028, no. 1, pp. 103--110, 2007.
\bibitem{b19} W. S. Guthrie and A. W. Birdsall, \textit{Effect of initial surface treatment timing on chloride concentrations in concrete bridge decks}, Tech. Rep., 2008.


\end{thebibliography}

\end{document}












\iffalse
The same domain and parameters are considered for FDM and FEM. Some of them are listed in Table~\ref{tab:model-parameters}.

%%%%%%Appendix
\begin{table}[!t]
\caption{Model parameters used in the chloride ingress simulations}
\label{tab:model-parameters}
\centering
\begin{tabular}{|l|l|}
\hline
\textbf{Symbol} & \textbf{Description} \\
\hline
$m_{0,\mathrm{init}}$ &
\begin{tabular}[c]{@{}l@{}}
Start month
\end{tabular} \\
\hline
$t_{\max}$ &
\begin{tabular}[c]{@{}l@{}}
Maximum number of days of exposure
\end{tabular} \\
\hline
$R_{BC}$ &
\begin{tabular}[c]{@{}l@{}}
Prescribed Dirichlet boundary condition of 0 or \\
prescribed Neumann boundary condition of 0
\end{tabular} \\
\hline
$d$ &
\begin{tabular}[c]{@{}l@{}}
Thickness of the specimen
\end{tabular} \\
\hline
$WC$ &
\begin{tabular}[c]{@{}l@{}}
Water--cement ratio
\end{tabular} \\
\hline
$hy$ &
\begin{tabular}[c]{@{}l@{}}
Degree of hydration
\end{tabular} \\
\hline
$VF$ &
\begin{tabular}[c]{@{}l@{}}
Volume fraction of aggregate
\end{tabular} \\
\hline
$AC$ &
\begin{tabular}[c]{@{}l@{}}
Air content (volume fraction)
\end{tabular} \\
\hline
$\kappa_\infty$ &
\begin{tabular}[c]{@{}l@{}}
Diffusivity for concrete
\end{tabular} \\
\hline
$D_i$ &
\begin{tabular}[c]{@{}l@{}}
Time-dependent diffusivity
\end{tabular} \\
\hline
$D_M$ &
\begin{tabular}[c]{@{}l@{}}
Time-dependent diffusivity coefficient
\end{tabular} \\
\hline
$\theta_i$ &
\begin{tabular}[c]{@{}l@{}}
Exterior temperature at month $i$ of exposure
\end{tabular} \\
\hline
$\theta_{\max}$ &
\begin{tabular}[c]{@{}l@{}}
Maximum temperature
\end{tabular} \\
\hline
$\theta_{\mathrm{cur}}$ &
\begin{tabular}[c]{@{}l@{}}
Temperature of current month
\end{tabular} \\
\hline
$M_i$ &
\begin{tabular}[c]{@{}l@{}}
Molarity at month $i$ of exposure
\end{tabular} \\
\hline
$M_{\mathrm{cur}}$ &
\begin{tabular}[c]{@{}l@{}}
Molarity of current month
\end{tabular} \\
\hline
$t_{\mathrm{cur}}$ &
\begin{tabular}[c]{@{}l@{}}
Time before exposure begins (s)
\end{tabular} \\
\hline
$\kappa_{\mathrm{rel}}^{\mathrm{surf}}$ &
\begin{tabular}[c]{@{}l@{}}
Relative diffusivity for surface layer
\end{tabular} \\
\hline
$d_{\mathrm{surf}}$ &
\begin{tabular}[c]{@{}l@{}}
Thickness of the surface layer
\end{tabular} \\
\hline
$E$ &
\begin{tabular}[c]{@{}l@{}}
Activation energy for chloride diffusion
\end{tabular} \\
\hline
$R$ &
\begin{tabular}[c]{@{}l@{}}
Universal gas constant: 0.008314 kJ/(mol K)
\end{tabular} \\
\hline
$t_{\mathrm{fac}}$ &
\begin{tabular}[c]{@{}l@{}}
Relative Arrhenius factor at curing time: \\
$e^{-E/(\theta_{\mathrm{cur}}R)} / e^{-E/(298.15R)}$
\end{tabular} \\
\hline
$t_{\max}$ &
\begin{tabular}[c]{@{}l@{}}
Maximal Arrhenius factor at $\theta_{\max}$ (K): \\
$e^{-E/(\theta_{\max}R)} / e^{-E/(298.15R)}$
\end{tabular} \\
\hline
$\alpha$ &
\begin{tabular}[c]{@{}l@{}}
Binding isotherm parameter~1
\end{tabular} \\
\hline
$\beta$ &
\begin{tabular}[c]{@{}l@{}}
Binding isotherm parameter~2
\end{tabular} \\
\hline
$k_{\mathrm{bind}}$ &
\begin{tabular}[c]{@{}l@{}}
Binding rate constant; capped at $0.2/\Delta t$ \\
to satisfy CFL stability for the reaction term
\end{tabular} \\
\hline
$C3A$ &
\begin{tabular}[c]{@{}l@{}}
C3A volume fraction (typically 5--10\%)
\end{tabular} \\
\hline
$C4AF$ &
\begin{tabular}[c]{@{}l@{}}
C4AF volume fraction (typically 5--13\%)
\end{tabular} \\
\hline
$C3A_{\mathrm{react}}$ &
\begin{tabular}[c]{@{}l@{}}
Formation of Friedel's salt from C3A (= 7.419)
\end{tabular} \\
\hline
$C4AF_{\mathrm{react}}$ &
\begin{tabular}[c]{@{}l@{}}
Formation of Friedel's salt from C4AF (= 4.119)
\end{tabular} \\
\hline
$k_{\mathrm{react}}$ &
\begin{tabular}[c]{@{}l@{}}
Chloride--aluminate reaction rate; capped at \\
$0.2/\Delta t$ to satisfy CFL stability
\end{tabular} \\
\hline
$t_{\mathrm{appl}}$ &
\begin{tabular}[c]{@{}l@{}}
Surface treatment application time
\end{tabular} \\
\hline
$r_{\mathrm{flags}}$ &
\begin{tabular}[c]{@{}l@{}}
Indicator for activation of reaction terms
\end{tabular} \\
\hline
$\gamma_{\mathrm{cem}}$ &
\begin{tabular}[c]{@{}l@{}}
Specific gravity of cement
\end{tabular} \\
\hline
$PF$ &
\begin{tabular}[c]{@{}l@{}}
Paste fraction $(1 - VF - AC)$
\end{tabular} \\
\hline
$Por$ &
\begin{tabular}[c]{@{}l@{}}
Pore liquid volume fraction: \\
$PF \cdot \dfrac{WC - 0.23\,hy}{WC + 1/\gamma_{\mathrm{cem}}}$ \\
(numerator clamped to be non-negative)
\end{tabular} \\
\hline
$MM_{\mathrm{Cl}}$ &
\begin{tabular}[c]{@{}l@{}}
Molecular mass of chloride
\end{tabular} \\
\hline
$\phi_0$ &
\begin{tabular}[c]{@{}l@{}}
Initial chloride concentration, modified as \\
$\phi_0 = \phi_0 \dfrac{PF \cdot 1000 \cdot \gamma_{\mathrm{cem}}}
{(1 + WC \gamma_{\mathrm{cem}})\, Por \, MM_{\mathrm{Cl}}}$
\end{tabular} \\
\hline
$\kappa_{\mathrm{conc}}$ &
\begin{tabular}[c]{@{}l@{}}
Diffusivity of concrete base layer: \\
$\kappa_\infty + D_i \, t_{\mathrm{cure}}^{-D_M}$
\end{tabular} \\
\hline
$\kappa_{\mathrm{skin}}$ &
\begin{tabular}[c]{@{}l@{}}
Diffusivity of surface layer: \\
$\kappa_{\mathrm{conc}} \cdot \kappa_{\mathrm{rel}}^{\mathrm{surf}}$
\end{tabular} \\
\hline
$\Delta d$ &
\begin{tabular}[c]{@{}l@{}}
Spatial element thickness
\end{tabular} \\
\hline
$\Delta t$ &
\begin{tabular}[c]{@{}l@{}}
Time step: $\Delta t = 0.2 (\Delta d)^2 /
\max\{\kappa_{\mathrm{conc}}, \kappa_{\mathrm{skin}}\}$ \\
(CFL condition accounting for aging and\\
temperature effects)\\
\end{tabular} \\
\hline
$Cl_{\max}$ &
\begin{tabular}[c]{@{}l@{}}
Maximum chloride binding capacity: \\
$\frac{(C3A \cdot C3A_{\mathrm{react}} + C4AF \cdot C4AF_{\mathrm{react}}}{(WC - 0.23\,hy)})$
\end{tabular} \\
\hline
\end{tabular}
\end{table}
\fi