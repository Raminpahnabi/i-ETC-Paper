\documentclass[conference]{IEEEtran}
\IEEEoverridecommandlockouts

\usepackage{placeins}

\usepackage{cite}
\usepackage{amsmath,amssymb,amsfonts}
\usepackage{algorithmic}
\usepackage{graphicx}
\usepackage{textcomp}
\usepackage{xcolor}
\newcommand{\Rd}[1]{{\color{red} #1}}
\def\BibTeX{{\rm B\kern-.05em{\sc i\kern-.025em b}\kern-08em
    T\kern-.1667em\lower.7ex\hbox{E}\kern-.125emX}}
    
%\usepackage{setspace}
%\doublespacing
\begin{document}


\title{Comparison of Finite Difference and Finite Element Methods for Modeling Chloride Diffusion in Concrete}

\author{\IEEEauthorblockN{ Ramin Pahnabi}
\IEEEauthorblockA{\textit{Civil and Construction Engineering} \\
\textit{Brigham Young University}\\
Provo, United States  \\
rpahnabi@byu.edu}
\and
\IEEEauthorblockN{W. Spencer Guthrie}
\IEEEauthorblockA{\textit{Civil and Construction Engineering} \\
\textit{Brigham Young University}\\
Provo, United States \\
guthrie@byu.edu}
\and
\IEEEauthorblockN{Kendrick M. Shepherd}
\IEEEauthorblockA{\textit{Civil and Construction Engineering} \\
\textit{Brigham Young University}\\
Provo, United States \\
kendrick\_shepherd@byu.edu}
}


\maketitle

\begin{abstract}
A proper understanding of chloride ion ingress into concrete and effective mitigation of its effects are crucial for preventing corrosion and premature failure of reinforced concrete structures.
Previous National Institute of Standards and Technology (NIST) software employed the finite difference method (FDM) to evaluate chloride diffusion in concrete; however, this approach is not readily extensible to modeling complex geometries including the effects of reinforcing rebar. The purpose of this study is to develop finite element method (FEM) software that can exactly reproduce the FDM results, and can be extended to handle more general and complex geometries, such as concrete containing reinforcing steel (rebar).
The FEM results show good agreement with the FDM results and the FEM could be used for the more complicated problem indicating little difference between both methods.
\end{abstract}

\begin{IEEEkeywords}
Chloride, Finite Deference Method (FDM), Finite Element Method (FEM), Concrete, Corrosion
\end{IEEEkeywords}

%%%%%%%%%%%%%%%%%%%%%%%%%%%%%%%%%%%%%%%%%%%%%%%%%%%%%%%%
\section{Introduction}

Chloride-induced corrosion is the main cause of deterioration of reinforced concrete (RC) structures in chloride-laden environments\cite{b1} and consequently, much research effort has been put towards the prediction of chloride-induced corrosion damage in RC structures\cite{b2,b3,b4,b5,b6,b7}. 
A considerable amount of research has been undertaken on the chloride-induced corrosion process, and different types of models, including empirical, mathematical, numerical and probabilistic ones, have been developed for service life prediction of corrosion-affected RC structures\cite{b2,b3,b8}.
In the past, both empirical and numerical approaches have been used to develop corrosion rate prediction models, each with its own strengths and shortcomings. Experimental models work well within the test conditions they were created for, but they often do not apply effectively to situations beyond the experiments\cite{b9,b10,b11}. 
Therefore, numerical methods are frequently used to ensure reliable prediction in general conditions. These methods include the finite difference method (FDM) and finite element method(FEM). 

National Institute of Standards and Technology(NIST) historically deployed FDM software\cite{b18,b19} used by state departments of transportation to predict chloride diffusion in concrete bridge decks\cite{b19}. While FDM provides a straightforward and computationally efficient method for modeling diffusion, it is often limited in handling complex geometries. 
There are many papers about the importance the FEM in chloride diffusion\cite{b12,b14}. Furthermore, studies \cite{b13}  show that the FEM can be used to predict crack initiation and propagation due to chloride-induced corrosion. These results are then used to update chloride propagation and predictions to realistically represent how cracking and corrosion affect each other. 

In this study, we leverage FEM's flexibility to simulate chloride diffusion and validate its performance against FDM results. 
While FDM is straightforward for simple domains, it struggles with the geometric complexity and boundary conditions. 
The variational formulation and capacity of FEM to operate on unstructured meshed make it inherently more flexible. 
The following sections demonstrate how FEM provides alternative framework for predicting chloride penetration, which should ultimately support geometrically precise durability evaluation and maintenance strategies for RC infrastructure.
Furthermore, a pathway is identified to translate the existing FDM code into the expanded FEM framework, enabling more accurate and flexible modeling of real-world structural conditions.


%%%%%%%%%%%%%%%%%%%%%%%%%%%%%%%%%%%%%%%%%%%%%%%%%%%%%%%%%
\section{Physics Problem}

\paragraph{Coupled system of equations}

Chloride ingress into concrete is often modeled using the coupled system of equations,
\begin{equation}
\left \{
\begin{aligned}
\frac{\partial C_{\mathrm{free}}}{\partial t}
&=
\frac{\partial}{\partial x}
\left(
D\,\frac{\partial C_{\mathrm{free}}}{\partial x}
\right)
-
k_{\rm react}\, r\left(
C_{\mathrm{react}}
\right)\, C_{\mathrm{free}}\\
&\quad - k_{\mathrm{bind}}
\left(
\frac{\alpha C_{\mathrm{free}}}{1 + \beta C_{\mathrm{free}}}
-
C_{\mathrm{ads}}
\right)
+
f, \\
&\quad 0 \leq x \leq L,\; 0 < t \leq T,
\end{aligned}
\right.
\label{eq:diff_react}
\end{equation}


\begin{equation}
\frac{dC_{\mathrm{react}}}{dt}
=
k_{\mathrm{react}}
r\left(
C_{\mathrm{react}}
\right)C_{\mathrm{free}},
\label{eq:reaction_rate}
\end{equation}


\begin{equation}
\frac{dC_{\mathrm{ads}}}{dt}
=
k_{\mathrm{bind}}
\left(
\frac{\alpha C_{\mathrm{free}}}{1 + \beta C_{\mathrm{free}}}
-
C_{\mathrm{ads}}
\right),
\label{eq:adsorption_rate}
\end{equation}

where $C_{\mathrm{free}}$ denotes the free chloride concentration in the pore solution, $C_{\mathrm{ads}}$ describes the adsorbed chloride concentration (mol/m$^3$), and $C_{\mathrm{react}}$ indicates chloride consumed due to irreversible chemical reaction.
Known inputs to this system include the diffusion coefficient, $D$; the reaction rate constant, $k_{\mathrm{react}}$; a reaction indicator function, $r$, that equals one in reactive regions and zero otherwise; a source term $f$; and coefficients $k_{\mathrm{bind}}$, $\alpha$, and $\beta$ which control the binding kinetics and adsorption capacity.
Here, $L$ denotes the length of the spatial domain and $T$ indicates the termination time.

Equation (\ref{eq:diff_react}) represents the diffusion--reaction model governing the free chloride concentration, which is inherently coupled with the amount of free chloride that is consumed due reactions and adsorption. 
Equations~(\ref{eq:reaction_rate}) and (\ref{eq:adsorption_rate}), consequently, represent irreversible reactions and Langmuir-type adsorption (which describes the binding of chloride ions to the concrete), respectively. 
By accounting for both the reaction and the adsorption terms, the bound chloride is separated from the total chloride content, and the remaining portion corresponds to the free chloride concentration.

\paragraph{Boundary and Initial Conditions}

The diffusion--reaction problem is closed by prescribing boundary conditions at both ends of the spatial domain. At the top boundary ($x=0$), the chloride concentration is typically fixed to a prescribed value, representing a constant or time--dependent exposure condition. This is enforced through a Dirichlet boundary condition,
\begin{equation}
U(0,t) = g,
\label{eq:bc_dirichlet}
\end{equation}
where $g$ denotes the imposed boundary concentration.

At the bottom boundary ($x=L$), chloride transport frequently is controlled through a specified outward flux, which is modeled using a Neumann boundary condition,
\begin{equation}
\left.
\frac{\partial U}{\partial x}
\right|_{x=L}
=
-\,h,
\label{eq:bc_neumann}
\end{equation}
where $h$ represents the prescribed flux at the boundary which is typically $0$. Together, these boundary conditions reflect a system in which chloride ingress is driven by exposure at one boundary while transport across the opposite boundary is regulated by a specified flux.

To solve the time-dependent diffusion–reaction problem, the initial condition must be specified before starting the time integration. 
In our case, the initial condition was set to zero throughout the domain, while a fixed chloride concentration g was prescribed at the top boundary section.


%%%%%%%%%%%%%%%%%%%%%%%%%%%%%%%%%%%%%%
\section{Numerical Methods}

The coupled diffusion--reaction--adsorption system is solved using a staggered (operator-splitting) time-integration scheme, in which diffusion and reaction/adsorption are solved sequentially at each time step~\cite{b17}. 
The Forward Euler method is used for all temporal discretization while we compute the spatial discretization using FDM or FEM.
Particularly, we use the Lie splitting \cite{b17} in which for a given time step, the reaction-diffusion system is solved first while neglecting the effects of adsorption, after which the adsorption equation is solved independently using the just-computed results from the reaction-diffusion system.
Finally, mass balance is restored by removing the computed adsorped chloride from the computed free chloride.
This splitting technique decouples the nonlinearity at the expense of being first order accurate in time. 
However, because Forward Euler is already first-order accurate in time, the effects of splitting mechanism are consistent with those of our integration scheme, leading overall efficiency.

\subsection{Finite Difference Method (FDM)}\label{AA}

 \paragraph{Spatial Discretization (Finite Difference)}

The spatial domain is discretized using a one-dimensional finite difference method. The interval $[0,L]$ is divided into $N$ uniform subintervals of size $\Delta x = L/N$, resulting in nodal points $x_i = i\,\Delta x$ for $i = 0,\dots,N$. 
The solution at each node is approximated by a discrete variable $d_i(t) \approx d(x_i,t)$.

Spatial derivatives in the diffusion term are approximated using second-order central differences.
Boundary conditions are incorporated directly into the discrete system. At the top boundary, the prescribed concentration is enforced by fixing the nodal value, while at the bottom boundary, the specified flux condition is imposed through a finite difference approximation of the spatial gradient. 

\paragraph{Time Discretization}

\textit{Forward Euler (Explicit) Method:} Using the Lie splitting, the Forward Euler method for the interior nodes ($1 \le i \le N-1$) gives


\begin{equation}
\begin{aligned}
d_i^{n+1} 
&= d_i^n 
+ \Delta t \Bigl[
f_i
-
\frac{
D_i (d_i^n - d_{i-1}^n)
-
D_{i+1} (d_{i+1}^n - d_i^n)
}{\Delta x^2}
\\
&\qquad
-
k_{\mathrm{react}}\, r_i\, d_i^n
\Bigr], 
\end{aligned}
\label{eq:forward_euler_interior}
\end{equation}



With boundary conditions applied explicitly:
At the top surface of the concrete slab, the chloride concentration is assumed to be maintained at a fixed value, representing continuous exposure to a chloride-rich environment (considered as Dirichlet BCs). This condition enforces a constant chloride level at the exposed surface throughout the simulation.
At the bottom of the slab, chloride transport is governed by a prescribed flux condition (Neumann BCs) or a constant chloride level (Dirichlet BCs). This represents controlled exchange across the lower boundary, such as limited chloride penetration into an underlying layer. By specifying the rate at which chloride enters or leaves the slab at this boundary, the model captures the physical constraint on chloride movement at the base of the domain.


Thus the explicit update at node $N$ is:
\begin{equation}
\begin{aligned}
d_N^{n+1}
&=
d_N^n
+
\Delta t
\Bigl[
f(x_N)
+
\frac{2h}{\Delta x}
+
\frac{2D_N}{\Delta x^2}
\bigl(d_{N-1}^n - d_N^n\bigr)
\\
&\qquad
-
k_{\mathrm{react}}\, r_N\, d_N^n
\Bigr],
\end{aligned}
\label{eq:forward_euler_boundary}
\end{equation}


\paragraph{CFL Condition for Explicit Scheme}\mbox{} To be sure that the explicit method (Forward-Euler) is reliable and the results  are stable, following limitation is considered:

\begin{equation}
\frac{\max(D) \, \Delta t}{\Delta x^2} \le 0.5 
\label{eq:cfl}
\end{equation}



%%%%%%%%%%%%%%%%%%%%%%%%%%%%%%%%%%%%%%%%%%%%%%%%%%%
%%%%%%%%%%%%%%%%%%%%%%%%% FEM %%%%%%%%%%%%%%%%%%%%%%%
%%%%%%%%%%%%%%%%%%%%%%%%%%%%%%%%%%%%%%%%%%%%%%%%%%%
\subsection{Finite Element Method (FEM)}\label{AA}

\paragraph{Spatial Discretization}

The one-dimensional diffusion--reaction problem is solved numerically using the
finite element method (FEM) with linear basis function. The formulation follows the standard Galerkin
approach described by \cite{b15}. FEM is particularly well suited for transport problems in heterogeneous materials such as concrete, as it naturally accommodates spatially varying material properties and allows for local mesh refinement \cite{b15}.

Making use of the Lie splitting, in abstract form, the weak problem can be written as: find $u \in S$ such that
\begin{equation}
(\dot{u},w) + (u, k_{react} r w)-a(u,w) = \ell(w)
\qquad \forall w \in V ,
\end{equation}

\begin{equation}
\begin{aligned}
a(u,w) &=
\int_{\Omega}
D\, u,_x\, w,_x \, dx,
\end{aligned}
\label{eq:nitsche-bilinear-reaction}
\end{equation}

\begin{equation}
	(u,w) = \int_{\Omega} u w dx,
\end{equation}

\begin{equation}
\begin{aligned}
\ell(w) &=
\int_{\Omega}
f\, w \, dx,
\end{aligned}
\label{eq:nitsche-linear-reaction}
\end{equation}


where $S$ and $V$ denote appropriate trial and test spaces, $a(\cdot,\cdot)$ is
the bilinear form associated with diffusion and reaction processes, $(\dot{u},w)$ represents the time flux, $(u, k_{react} r w)$ decribes the reaction terms, and 
$\ell(\cdot)$ represents the contribution of source terms and boundary fluxes \cite{b15}.

The solution field is approximated within a finite-dimensional subspace spanned
by shape functions defined over a partition of the spatial domain.
This leads to a discrete representation of the form:
\begin{equation}
u_h(x) = \sum_{i=1}^{N} d_i(t)\, N_i(x),
\end{equation}
where $N_i(x)$ are the finite element shape functions and $d_i$ are the unknown
nodal values(coefficients). Substituting this approximation into the weak form and choosing the test functions from the same space yields a system of algebraic equations of the form.


\paragraph{Time Discretization}

For time-dependent problems, the semi-discrete finite element formulation
leads to a system of ordinary differential equations of the form
\begin{equation}
\mathbf{M}\, \dot{\mathbf{d}}(t)
+
(\mathbf{K} + \mathbf{R})\, \mathbf{d}(t)
=
\mathbf{F}(t),
\label{eq:semi-discrete}
\end{equation}
where $\mathbf{M}$ is the mass matrix, $\mathbf{K}$ is the diffusion stiffness
matrix, $\mathbf{R}$ is the reaction matrix, $\mathbf{d}(t)$ is the vector of
time-dependent nodal unknowns, and $\mathbf{F}(t)$ is the load vector.


\textit{Forward Euler Scheme}: Using a Forward Euler discretization in time, the time derivative in
\eqref{eq:semi-discrete} is approximated as
\[
\dot{\mathbf{d}}(t^n)
\approx
\frac{\mathbf{d}^{n+1} - \mathbf{d}^n}{\Delta t},
\]
where $\Delta t$ denotes the time-step size and the superscript $n$ indicates
the discrete time level.

Substituting this approximation into \eqref{eq:semi-discrete} and solving for
$\mathbf{d}^{n+1}$ yields the explicit update formula
\begin{equation}
\mathbf{d}^{n+1}
=
\mathbf{d}^n
+
\Delta t \, \mathbf{M}^{-1}
\big(
\mathbf{F}^n
-
(\mathbf{K} + \mathbf{R})\, \mathbf{d}^n
\big),
\label{eq:forward-euler}
\end{equation}
which advances the solution from time level $t^n$ to $t^{n+1}$.

Equation \eqref{eq:forward-euler} is conditionally stable and requires the time
step $\Delta t$ to satisfy a problem-dependent stability constraint \eqref{eq:cfl}.


%%%%%%%%%%%%%%%%%%%%%%%%%%%NITSCHE%%%%%%%%%%%%%%%%%%%%%%%
%%%%%%%%%%%%%%%%%%%%%%%%%%%NITSCHE%%%%%%%%%%%%%%%%%%%%%%%
%%%%%%%%%%%%%%%%%%%%%%%%%%%NITSCHE%%%%%%%%%%%%%%%%%%%%%%%
\subsection{Nitsche Method for Weak Imposition of Dirichlet Boundary Conditions}

To weakly impose essential boundary conditions within the finite element
framework as is done in the FDM software, Nitsche’s method is employed \cite{b16}. This approach allows Dirichlet boundary conditions to be enforced variationally, while maintaining consistency and stability of the numerical scheme. In contrast, finite difference methods (FDM) apply boundary conditions directly at grid points, making their implementation simple, especially for structured domains.

In Nitsche’s method \cite{b16}, the standard weak formulation is augmented by additional boundary terms that weakly enforce the Dirichlet condition. These terms ensure consistency with the strong form, restore symmetry of the bilinear form, and add a stabilization contribution controlled by a penalty parameter.


The resulting problem is
%
\begin{equation}
 	(\dot{u},w) + (u,k_{react}rw) - a_{\rm Nitsche}(u,w) = \ell_{\rm Nitsche}(w)
\end{equation}
where the bilinear form is
%
\begin{equation}
\begin{aligned}
a_{\rm Nitsche}(u,w)
&=
a(u,w) -
\int_{\Gamma_D}
D\, u,_x\, w \, ds\\
&\quad
-
\int_{\Gamma_D}
D\, w,_x\, u \, ds
+
\int_{\Gamma_D}
\frac{\gamma D}{h_e}\, u\, w \, ds,
\end{aligned}
\label{eq:nitsche-bilinear-reaction}
\end{equation}
where $u$ is the trial function, $w$ is the test function, $h_e$ is a
characteristic boundary element length, and $\gamma > 0$ is a penalty parameter. A value of $\gamma = 100$ was used, which yielded results in agreement with the NIST software.
The associated linear functional is given by
%
\begin{equation}
\begin{aligned}
\ell_{\rm Nitsche}(w)
&=
\ell(w) + 
\int_{\Gamma_D}
D\, w,_x\, g \, ds
+
\int_{\Gamma_D}
\frac{\gamma D}{h_e}\, g\, w \, ds,
\end{aligned}
\label{eq:nitsche-linear-reaction}
\end{equation}
where $f$ is the body force and $g$ denotes the prescribed Dirichlet boundary
value. 



%NEW IMAGES
\section{Validation and Comparison}

In this section, we compare the finite difference method (FDM) with the finite element formulations. 
The objectives are threefold: first, to verify that all methods converge; second, to numerically assess the error associated with the coarse mesh; and third, to examine the differences between the FEM formulations and the FDM commonly used in practice.
To make these comparisons, we compute chloride diffusion through a slab studied in \cite{} \Rd{Dr. Guthrie's paper}. 
A complete discussion of concrete-chloride chemistry present in these simulations is outside the scope of this paper, but input parameters and forcing behavior for these simulations are described in Tables \ref{tab:model-parameters} and \ref{tab:monthly-data}. 
The interested reader is referred to the work of Bentz and Feng \cite{b19} for additional guidance.


%%%%%%Appendix
\begin{table}[!tb]
\caption{Model parameters used in the chloride ingress simulations}
\label{tab:model-parameters}
\centering
\begin{tabular}{|l|l|}
\hline
\textbf{Symbol} & \textbf{Description} \\
\hline
$t_{\max}$ &
\begin{tabular}[c]{@{}l@{}}
Maximum number of days of exposure: 1800 days
\end{tabular} \\
\hline
$d$ &
\begin{tabular}[c]{@{}l@{}}
Thickness of the specimen: 0.25 m
\end{tabular} \\
\hline
$WC$ &
\begin{tabular}[c]{@{}l@{}}
Water--cement ratio: 0.5
\end{tabular} \\
\hline
$hy$ &
\begin{tabular}[c]{@{}l@{}}
Degree of hydration: 0.75
\end{tabular} \\
\hline
$VF$ &
\begin{tabular}[c]{@{}l@{}}
Volume fraction of aggregate: 70\%
\end{tabular} \\
\hline
$AC$ &
\begin{tabular}[c]{@{}l@{}}
Air content (volume fraction): 2.0\%
\end{tabular} \\
\hline
$D_i$ &
\begin{tabular}[c]{@{}l@{}}
Time-dependent diffusivity: 6e-12
\end{tabular} \\
\hline
$\theta_i$ &
\begin{tabular}[c]{@{}l@{}}
Exterior temperature at month $i$ of exposure:
Table~\ref{tab:monthly-data}.
\end{tabular} \\


\hline
$t_{\mathrm{cur}}$ &
\begin{tabular}[c]{@{}l@{}}
Time before exposure begins (s): 28 days
\end{tabular} \\
\hline
$\kappa_{\mathrm{rel}}^{\mathrm{surf}}$ &
\begin{tabular}[c]{@{}l@{}}
Relative diffusivity for surface layer: 1
\end{tabular} \\
\hline
$d_{\mathrm{surf}}$ &
\begin{tabular}[c]{@{}l@{}}
Thickness of the surface layer: 5.0 mm
\end{tabular} \\
\hline
$E$ &
\begin{tabular}[c]{@{}l@{}}
Activation energy for chloride diffusion: 40 kJ/mole 
\end{tabular} \\
\hline
$R$ &
\begin{tabular}[c]{@{}l@{}}
Universal gas constant: 0.008314 kJ/(mol K)
\end{tabular} \\


\hline
$\alpha$ &
\begin{tabular}[c]{@{}l@{}}
Binding isotherm parameter: 1.67
\end{tabular} \\
\hline
$\beta$ &
\begin{tabular}[c]{@{}l@{}}
Binding isotherm parameter: 4.08
\end{tabular} \\


\hline
$k_{\mathrm{bind}}$ &
\begin{tabular}[c]{@{}l@{}}
Binding rate constant: 1e-07
\end{tabular} \\
\hline
$C3A$ &
\begin{tabular}[c]{@{}l@{}}
C3A volume fraction (typically 5--10\%)
\end{tabular} \\
\hline
$C4AF$ &
\begin{tabular}[c]{@{}l@{}}
C4AF volume fraction (typically 5--13\%)
\end{tabular} \\
\hline
$C3A_{\mathrm{react}}$ &
\begin{tabular}[c]{@{}l@{}}
Formation of Friedel's salt from C3A (= 7.419)
\end{tabular} \\
\hline
$C4AF_{\mathrm{react}}$ &
\begin{tabular}[c]{@{}l@{}}
Formation of Friedel's salt from C4AF (= 4.119)
\end{tabular} \\


\hline
$k_{\mathrm{react}}$ &
\begin{tabular}[c]{@{}l@{}}
Chloride--aluminate reaction rate: 1e-08
\end{tabular} \\
\hline

\end{tabular}
\end{table}

\begin{table}[!tb]
\caption{Monthly chloride concentration and temperature values used in the simulation.}
\label{tab:monthly-data}
\centering
\begin{tabular}{|l|c|c|}
\hline
\textbf{Month} & \textbf{Cl$^{-}$} & \textbf{Temperature (K)} \\
\hline
0  & 4.0000 & 278.15 \\
\hline
1  & 4.0000 & 278.15 \\
\hline
2  & 4.0000 & 288.15 \\
\hline
3  & 4.0000 & 288.15 \\
\hline
4  & 4.0000 & 293.15 \\
\hline
5  & 4.0000 & 298.15 \\
\hline
6  & 4.0000 & 303.15 \\
\hline
7  & 4.0000 & 303.15 \\
\hline
8  & 4.0000 & 298.15 \\
\hline
9  & 4.0000 & 293.15 \\
\hline
10 & 4.0000 & 288.15 \\
\hline
11 & 4.0000 & 283.15 \\
\hline
\end{tabular}
\end{table}


%With Finite Element method and without the use of the Mass lumping, the finite element results were found to be in close agreement with those obtained using the finite difference method (FDM). 
%
%When the FEM is combined with mass lumping, the resulting finite element solution reproduces the FDM results exactly. 
%This agreement provides confidence that the proposed numerical techniques do not alter the underlying physics of the problem and remain consistent with the physics-based simulation
%benchmarks defined by the National Institute of Standards and Technology (NIST) software.

Because of the numerous types of FEM computations performed, the following notation is followed hereafter for FEM: a plus sign (+) indicates that the corresponding technique is used, while a minus sign (-) indicates that it is not, with potential methods being mass lumping (ML) and Nitsche (N).
For instance, FEM(-ML +N) indicates that no  mass lumping was used but that Nitsche’s method was used. 
%Using this notation, the FEM variants are as follows:  
%\begin{itemize}
%    \item FEM(-ML -N): without mass lumping and without Nitsche’s method  
%    \item FEM(-ML +N): without mass lumping but with Nitsche’s method  
%    \item FEM(+ML -N): with mass lumping but without Nitsche’s method  
%    \item FEM(+ML +N): with mass lumping and with Nitsche’s method
%\end{itemize}

Comparative plots are first presented for FDM, FEM(-ML -N), and FEM(+ML -N) because each of these strongly enforces input Dirichlet boundary conditions.
Coarse mesh computations involved a uniform spatial discretization with 100 spatial  elements and 500 time steps Fig.~\ref{fig:fdm-vs-fem-nitsche_100},
while computational on a finer discretization involving 500 spatial elements and 100,000 time steps are shown in Fig.~\ref{fig:fdm-vs-fem-nitsche_500}.
Differences between computed FEM values compared to the FDM benchmark are presented in Figs~\ref{fig:fdm-vs-fem-error-all-coarse} and \ref{fig:fdm-vs-fem-error-all-fine} for both the coarse and the fine meshes, respectively.
As is expected, FEM(+ML -N) results in exactly the same values as the FDM solver, while a small difference appears when mass lumping is not used.
However, with refinement, this difference diminishes.

\begin{figure}[!b]
\centering
\includegraphics[width=\linewidth]{100_plots.png}
\caption{Comparing FDM with FEM with Nitsche and ML with coarse mesh.}
\label{fig:fdm-vs-fem-nitsche_100}
\end{figure}

\begin{figure}[!tb]
\centering
\includegraphics[width=\linewidth]{500_plots.png}
\caption{Comparing FDM with FEM with Nitsche and ML with fine mesh.}
\label{fig:fdm-vs-fem-nitsche_500}
\end{figure}


\begin{figure}[!tb]
\centering
\includegraphics[width=\linewidth]{100_error.png}
\caption{Error comparison between FDM and FEM formulations on the coarse mesh.}
\label{fig:fdm-vs-fem-error-all-coarse}
\end{figure}

\begin{figure}[!tb]
\centering
\includegraphics[width=\linewidth]{500_error.png}
\caption{Error comparison between FDM and FEM formulations on the fine mesh.}
\label{fig:fdm-vs-fem-error-all-fine}
\end{figure}

Also in Fig.~\ref{fig:coarse_vs_fine}, we compare the results from the fine and coarse meshes in a single plot. In Fig.~\ref{fig:coarse_vs_fine}, you can see that with refinement, we obtain convergence with the FEM method by comparing the coarse and fine mesh results, which gives us confidence in the results. Based on this plot, the FEM(+ML -N) results are one order of magnitude smaller than the others.
\Rd{Also include a plot that describes how FEM -N on coarse scale compares to FDM on fine scale and discuss confidence in our results.}

\begin{figure}[!tb]
\centering
\includegraphics[width=\linewidth]{coarse_vs_fine_error.png}
\caption{Error comparison between FDM and FEM formulations on coarse and fine mesh}
\label{fig:coarse_vs_fine}
\end{figure}


\begin{figure}[!tb]
\centering
\includegraphics[width=\linewidth]{100_N_plots.png}
\caption{Error comparison between FDM and FEM formulations}
\label{fig:nitsche_plots}
\end{figure}

Conversely, results using the Nitsche method differ more dramatically than those with strong boundary enforcement, and are shown in Fig.~\ref{fig:nitsche_plots}.
As can be seen, variation in the boundary enforcement parameter $\gamma$ yields different computed solutions, with higher values of $\gamma$ more strongly enforcing the top boundary constraint, but also introducing worse matrix conditioning and smaller stable time steps.
Simulations performed here use 100 uniform spatial elements and 10{,}000 time steps.
The absolute difference between FDM results, each of the Nitsche methods, and the FEM results with strongly enforced boundary conditions are shown in  Fig.~\ref{fig:nitsche_error_with_fem}.
Particularly, Nitsche methods yield results whose difference compared to the FDM results are an order of magnitude larger than that of the FEM results wherein Dirichlet boundary conditions are strongly enforced.


Nitsche methods are also particularly interesting because stability in the method can still yield non-monotonic solutions, whereas all other solutions are monotonic.
As is illustrated in \ref{fig:nitsche_plots}, if $\gamma \geq 80$ is chosen, the maximal chloride value is computed as larger than the input value, making the results less physically meaningful.
Consequently, care should be taken when using the Nitsche method to ensure that results are interpreted correctly and match expected concentrations compared to physical measurement.

%\begin{figure}[!tb]
%\centering
%\includegraphics[width=\linewidth]{100_N_error.png}
%\caption{Error comparison between FDM and FEM formulations with Nitsche's method}
%\label{fig:nitsche_error}
%\end{figure}



%Fig.~\ref{fig:nitsche_error} illustrates the difference between the Nitsche's method results and the NIST software results.
%In this section, we considered fixed time steps and spatial discretization. The spatial discretization was set to 100 elements, and the number of time steps was 10{,}000. 
%We varied the penalty parameter $\gamma$ to investigate its effect on the results.
%
%As shown in Fig.~\ref{fig:nitsche_error} , increasing the value of $\gamma$ improved the accuracy up to $\gamma = 50$, which provided the best performance among the selected values. 
%However, for $\gamma > 50$, the results gradually deteriorated. Furthermore, due to stability limitations, values of $\gamma$ greater than 80 led to unstable and nonphysical results.

\begin{figure}[!tb]
\centering
\includegraphics[width=\linewidth]{100_N_FEM_error.png}
\caption{Error comparison between FDM and selected FEM formulations.}
\label{fig:nitsche_error_with_fem}
\end{figure}


%
%Fig.~\ref{fig:nitsche_error_with_fem} presents a comparison between FEM(+ML $-$N), FEM($-$ML $-$N), and FEM(+ML +N) against the FDM solution. 
%It can be observed that the best performance corresponds to FEM(+ML $-$N). 


%%%%%%
\iffalse
\begin{table}[!tb]
\caption{Convergence between coarse and fine meshes: maximum and mean absolute difference of the solution.}
\label{tab:mesh-convergence}
\centering
\begin{tabular}{|l|c|c|}
\hline
\textbf{Method} & \textbf{Max.\ abs.\ diff.} & \textbf{Mean abs.\ diff.} \\
\hline
FDM & 0.0389 & 0.00412 \\
\hline
FEM(-ML -N) & 0.0445 & 0.00569 \\
\hline
FEM(+ML -N) & 0.0440 & 0.00475 \\
\hline
FEM(-ML +N) & 0.201 & 0.0281 \\
\hline
FEM(+ML +N) & 0.0389 & 0.00412 \\
\hline
\end{tabular}
\end{table}

Table~\ref{tab:mesh-convergence} reports the maximum and mean absolute differences between the fine-mesh solution and the coarse-mesh solution (interpolated onto the fine grid).
All methods exhibit convergence with mesh refinement.
For FDM and the standard FEM formulations (with or without mass lumping), the discrepancies are of order $10^{-2}$ in the maximum norm and about $5\times10^{-3}$ on average, indicating that the coarse mesh already provides a reasonably accurate approximation.

A larger difference is observed for the FEM formulation using Nitsche's method without mass lumping (maximum difference $\approx 0.20$, mean $\approx 0.028$).
This indicates a stronger sensitivity of this formulation to mesh resolution.
When mass lumping is combined with Nitsche's method, however, the differences reduce to the same level as the other methods.
Overall, these results confirm the convergence of all approaches while highlighting the improved stability and mesh robustness obtained by combining Nitsche's method with mass lumping.
\fi


The same domain and parameters are considered for FDM and FEM. Some of them are listed in Table~\ref{tab:model-parameters}.

\section{Discussion}

\subsection{Advantages of FEM over FDM}

The finite element method (FEM) offers several advantages over the finite difference method (FDM), particularly for engineering applications including complex geometries and boundary conditions. Unlike standard FDM, which is typically restricted to structured grids and regular domains, FEM can naturally accommodate irregular geometries and spatially varying material properties. In addition, FEM provides greater flexibility in the treatment of complex boundary conditions and is generally more suitable for solving problems involving complicated physical processes. 
As a result, complex geometric problems can often be modeled more effectively using FEM than FDM while still producing the same values in geometrically simple cases.
Results also indicate that evaluations using Nitsche's method should be treated with care because a poor choice of the parameter $\gamma$ may lead to overly weak enforcement of boundary conditions or may lead to non-physical, non-monotone results.

\subsection{Limitations}

Despite their effectiveness, both FEM and FDM require careful treatment of boundary conditions and discretization parameters to ensure numerical accuracy and stability. The FDM is computationally efficient and simple to implement, but its applicability is limited for irregular domains and complex geometries. On the other hand, while FEM provides greater flexibility and modeling capability, it typically involves higher computational cost and increased implementation complexity compared to FDM.

\section{Conclusions}

In this study, the finite element method (FEM) was applied to simulate one-dimensional chloride diffusion and its performance was compared with the finite difference method (FDM). The results show that, with appropriate boundary treatment, such as the use of Nitsche's method and mass lumping, the FEM solutions closely reproduce the reference FDM results. The results of the FEM(+ML +N) are still comparable for the considered diffusion problem. 

In addition to its accuracy, FEM offers greater flexibility for modeling complex engineering problems in reinforced concrete. In particular, it allows the treatment of irregular geometries, spatially varying material properties, and more sophisticated boundary conditions, which are difficult to handle using standard FDM approaches.

Future work could extend FEM analysis for chloride diffusion for 2D and 3D analysis, alternative temporal discretization, or multiphysics extensions to cracking or advective flow.

\section{Acknowledgements}
W. S. Guthrie and R. Pahnabi were partially funded by the King Hussein Fellowship.
This financial support is gratefully acknowledged.

\begin{thebibliography}{00}
\bibitem{b1}
M. Otieno, H. Beushausen, and M. Alexander, ``Chloride-induced corrosion of steel in cracked concrete—Part II: Corrosion rate prediction models," Cement and Concrete Research, Vol. 41, No. 4, pp. 443--452, 2011.
\bibitem{b2}U. M. Angst, E. Rossi, C. Boschmann K\"athler, D. Mannes, P. Trtik,
B. Elsener, Z. Zhou, and M. Strobl, ``Chloride-induced corrosion of steel in concrete—insights from bimodal neutron and X-ray microtomography combined with ex-situ microscopy," Materials and Structures, 2024.
\bibitem{b3}M. Otieno, H. Beushausen, and M. Alexander, ``Chloride-induced corrosion of steel in cracked concrete—Part I: Experimental studies under accelerated and natural marine environments,"
Cement and Concrete Research, 2015.
\bibitem{b4}
C. S. Das, H. Zheng, and J.-G. Dai, ``A review of chloride-induced steel corrosion in coastal reinforced concrete structures: Influence of micro-climate," Construction and Building Materials, 2024.
\bibitem{b5}
C. Liang, Z. Cai, H. Wu, J. Xiao, Y. Zhang, and Z. Ma, ``Chloride transport and induced steel corrosion in recycled aggregate concrete: A review," Construction and Building Materials, 2021.
\bibitem{b6}J. N\v{e}me\v{c}ek, P. Tr\'avn\'i\v{c}ek, J. N\v{e}me\v{c}kov\'a, and J. Kruis, ``Mitigation of chloride induced corrosion in reinforced concrete structures and its modeling," Computers and Concrete, 2021.  
\bibitem{b7}
S. C. Paul and G. P. A. Greeff van Zijl, ``Corrosion deterioration of steel in cracked SHCC," International Journal of Concrete Structures and Materials, 2017.
\bibitem{b8} J. Zhang and M. M. S. Cheung, ``Modeling of chloride-induced corrosion in reinforced concrete structures," Materials and Structures, Vol. 45, No. 10, pp. 1555--1566, 2012.
\bibitem{b9} P. Li, C. Li, C. Jia, and D. Li, ``A comparative study on chloride diffusion in concrete exposed to different marine environment conditions," Cement and Concrete Composites, 2024.  
\bibitem{b10} P. Su, Q. Dai, and E. S. Kane, ``Predicting chloride ingression in concrete containing different SCMs based on chloride binding and electrical resistivity," Construction and Building Materials, 2024.  
\bibitem{b11} T. Ferenc, E. Wojtczak, B. Meronk, J. Ryl, K. Wilde, and M. Rucka, ``Characterization of corrosion-induced fracture in reinforced concrete beams using electrical potential, ultrasound and low-frequency vibration," Construction and Building Materials, 2023. 
\bibitem{b12}J. Xiao, J. Ying, and L. Shen, ``FEM simulation of chloride diffusion in modeled recycled aggregate concrete," Construction and Building Materials, Vol.~29, pp.~12--23, 2012. 
\bibitem{b13}
E. Redaelli, L. Bertolini, W. Peelen, and R. Polder, ``FEM-models for the propagation period of chloride induced reinforcement corrosion," Materials and Corrosion, Vol. 57, No. 8, pp. 628--635, 2006.
\bibitem{b14} E. Korec, M. Jir\'asek, H. S. Wong, and E. Mart\'inez-Pa\~neda, ``A phase-field chemo-mechanical model for corrosion-induced cracking in reinforced concrete," Mechanics of Materials, 2023.
\bibitem{b15} T.~J.~R. Hughes, The Finite Element Method: Linear Static and Dynamic Finite Element Analysis. Upper Saddle River, NJ: Prentice Hall, 2000.
\bibitem{b16} E.~Burman, ``A consistent Nitsche formulation for the weak imposition
of boundary conditions in convection--diffusion problems,'' Computer Methods in Applied Mechanics and Engineering, vol.~199, no.~47--48, pp.~2845--2855, 2010.
\bibitem{b17} D.~Lanser and J.~G.~Verwer, ``Analysis of operator splitting for advection--diffusion--reaction problems from air pollution modelling,'' Journal of Computational and Applied Mathematics, vol.~111, no.~1--2, pp.~201--216, 1999.
\bibitem{b18} A. W. Birdsall, W. S. Guthrie, and D. P. Bentz, ``Effects of initial surface treatment timing on chloride concentrations in concrete bridge decks,'' Transp. Res. Rec., vol. 2028, no. 1, pp. 103--110, 2007.
\bibitem{b19} W. S. Guthrie and A. W. Birdsall, \textit{Effect of initial surface treatment timing on chloride concentrations in concrete bridge decks}, Tech. Rep., 2008.
\bibitem{b19} P. Dale, F. X. Bentz, and R. D. Hooton, ``Time-dependent diffusivities: Possible misinterpretation due to spatial dependence,'' in Proc. 2nd Int. RILEM Workshop on Testing and Modelling the Chloride Ingress into Concrete, Paris, 2000.


\end{thebibliography}

\end{document}

